%% -----------------------------------------------------------------------------
%% MIT License
%%
%% Copyright 2022 ThirtySomething
%%
%% Permission is hereby granted, free of charge, to any person obtaining a
%% copy of this software and associated documentation files (the "Software"),
%% to deal in the Software without restriction, including without limitation
%% the rights to use, copy, modify, merge, publish, distribute, sublicense,
%% and/or sell copies of the Software, and to permit persons to whom the
%% Software is furnished to do so, subject to the following conditions:
%%
%% The above copyright notice and this permission notice shall be included
%% in all copies or substantial portions of the Software.
%%
%% THE SOFTWARE IS PROVIDED "AS IS", WITHOUT WARRANTY OF ANY KIND, EXPRESS
%% OR IMPLIED, INCLUDING BUT NOT LIMITED TO THE WARRANTIES OF MERCHANTABILITY,
%% FITNESS FOR A PARTICULAR PURPOSE AND NONINFRINGEMENT. IN NO EVENT SHALL
%% THE AUTHORS OR COPYRIGHT HOLDERS BE LIABLE FOR ANY CLAIM, DAMAGES OR OTHER
%% LIABILITY, WHETHER IN AN ACTION OF CONTRACT, TORT OR OTHERWISE, ARISING
%% FROM, OUT OF OR IN CONNECTION WITH THE SOFTWARE OR THE USE OR OTHER
%% DEALINGS IN THE SOFTWARE.
%% -----------------------------------------------------------------------------

\section{Commands}
\label{sec:Commands}

To simplify things for newcomers, various commands have been defined. The use
of the commands is optional. However, using them makes it clearer what the
author wants to achieve with the corresponding command. It is recommended to
end commands without parameters with \tsTextMonospace{\{\}}, even if this is
not necessary. However, it leads to the fact that, for example, after such a
command there can also be a space character -- otherwise this is removed by
\LaTeX{} when compiling.

\begin{footnotesize}
    \renewcommand*{\arraystretch}{1.5}
    \begin{longtable}{ | P{0.53\linewidth} | p{0.4\linewidth} | }
        \hline
        \tsTextBold{Command definition}                                                                    &
        \tsTextBold{Meaning}                                                                                 \\
        \hline
        % ----------------------------------------------------------------------
        \tsTextMonospace{\tsBackslash{}tsAppendixIncludeFile\{Filename\}}                                  &
        This command will add an appendix section. The section consists of a
        \tsTextMonospace{tsTextMonospace} section header of \tsTextMonospace{Filename}
        and will include the file \tsTextMonospace{Filename} in a verbatim
        environment. See appendix \ref{appendix:README.md} how it looks like.                                \\
        \hline
        % ----------------------------------------------------------------------
        \tsTextMonospace{\tsBackslash{}tsAppendixSection\{Appendix\}}                                      &
        This command will add an appendix, also a so-called label. Thus the
        appendix can be referenced via
        \tsTextMonospace{\tsBackslash{}ref\{appendix:Appendix\}}
        or via
        \tsTextMonospace{\tsBackslash{}nameref\{appendix:Appendix\}}
        somewhere else.                                                                                      \\
        \hline
        % ----------------------------------------------------------------------
        \tsTextMonospace{\tsBackslash{}tsArrowRight\{\}}                                                   &
        \tsArrowRight{}                                                                                      \\
        \hline
        % ----------------------------------------------------------------------
        \tsTextMonospace{\tsBackslash{}tsArrowRightDouble\{\}}                                             &
        \tsArrowRightDouble{}                                                                                \\
        \hline
        % ----------------------------------------------------------------------
        \tsTextMonospace{\tsBackslash{}tsArrowDown\{\}}                                                    &
        \tsArrowDown{}                                                                                       \\
        \hline
        % ----------------------------------------------------------------------
        \tsTextMonospace{\tsBackslash{}tsArrowDownDouble\{\}}                                              &
        \tsArrowDownDouble{}                                                                                 \\
        \hline
        % ----------------------------------------------------------------------
        \tsTextMonospace{\tsBackslash{}tsArrowLeft\{\}}                                                    &
        \tsArrowLeft{}                                                                                       \\
        \hline
        % ----------------------------------------------------------------------
        \tsTextMonospace{\tsBackslash{}tsArrowLeftDouble\{\}}                                              &
        \tsArrowLeftDouble{}                                                                                 \\
        \hline
        % ----------------------------------------------------------------------
        \tsTextMonospace{\tsBackslash{}tsArrowUp\{\}}                                                      &
        \tsArrowUp{}                                                                                         \\
        \hline
        % ----------------------------------------------------------------------
        \tsTextMonospace{\tsBackslash{}tsArrowUpDouble\{\}}                                                &
        \tsArrowUpDouble{}                                                                                   \\
        \hline
        % ----------------------------------------------------------------------
        \tsTextMonospace{\tsBackslash{}tsBackslash\{\}}                                                    &
        \tsBackslash{}                                                                                       \\
        \hline
        % ----------------------------------------------------------------------
        \tsTextMonospace{\tsBackslash{}tsBullet\{\}}                                                       &
        \tsBullet{}                                                                                          \\
        \hline
        % ----------------------------------------------------------------------
        \tsTextMonospace{\tsBackslash{}tsCaptionLabelFigure\{Figure\}}                                     &
        This command is usually used below a mapping. There is not only a
        description added, but also a so-called label. Thus the figure can be
        referenced via
        \tsTextMonospace{\tsBackslash{}ref\{fig:Figure\}}
        or via
        \tsTextMonospace{\tsBackslash{}nameref\{fig:Figure\}}
        somewhere else.                                                                                      \\
        \hline
        % ----------------------------------------------------------------------
        \tsTextMonospace{\tsBackslash{}tsCaptionLabelFormula\{Formula\}}                                   &
        This command is similar to
        \tsTextMonospace{\tsBackslash{}tsCaptionLabelFigure},
        but is referenced via
        \tsTextMonospace{\tsBackslash{}ref\{for:Formula\}} or via
        \tsTextMonospace{\tsBackslash{}nameref\{for:Formula\}}.                                              \\
        \hline
        % ----------------------------------------------------------------------
        \tsTextMonospace{\tsBackslash{}tsCaptionLabelTable\{Table\}}                                       &
        This command is similar to
        \tsTextMonospace{\tsBackslash{}tsCaptionLabelFigure}, but is
        referenced via
        \tsTextMonospace{\tsBackslash{}ref\{tab:Table\}} or via
        \tsTextMonospace{\tsBackslash{}nameref\{tab:Table\}}.                                                \\
        \hline
        % ----------------------------------------------------------------------
        \tsTextMonospace{\tsBackslash{}tsCitation\{ref\}}                                                  &
        References citation in bibliography, see also \nameref{subsec:Bibliography references} for details.  \\
        \hline
        % ----------------------------------------------------------------------
        \tsTextMonospace{\tsBackslash{}tsDegree\{zahl\}}                                                   &
        Without number \tsDegree{} or with number \tsDegree{715}!                                            \\
        \hline
        % ----------------------------------------------------------------------
        \tsTextMonospace{\tsBackslash{}tsDiameter\{\}}                                                     &
        \tsDiameter{}                                                                                        \\
        \hline
        % ----------------------------------------------------------------------
        \tsTextMonospace{\tsBackslash{}tsFootnoteDef\{text\}\{ref\}}                                       &
        Defines a footnote\tsFootnoteDef{I am a footnote.}{fndef} with
        a label, which can be referenced elsewhere using
        \tsTextMonospace{\tsBackslash{}tsFootnoteRef\{ref\}}.                                                \\
        \hline
        % ----------------------------------------------------------------------
        \tsTextMonospace{\tsBackslash{}tsFrac\{num1\}\{num2\}}                                             &
        Defines a frac like \tsFrac{1}{2}.                                                                   \\
        \hline
        % ----------------------------------------------------------------------
        \tsTextMonospace{\tsBackslash{}tsFootnoteRef\{ref\}}                                               &
        References a footnote\tsFootnoteRef{fndef} which was defined elsewhere
        using \tsTextMonospace{\tsBackslash{}tsFootnoteDef\{text\}\{ref\}}.                                  \\
        \hline
        % ----------------------------------------------------------------------
        \tsTextMonospace{\tsBackslash{}tsImage\{filename\}\{description\}\newline\{options\}}              &
        With this, an image can be included -- it then automatically gets a
        description. It is mandatory to specify three parameter brackets.
        The first parameter is the file name. The second parameter is the
        signature under the image. The third parameter, the options, are
        optional, that is, the brackets can be empty. These are the same
        parameters that can be used with
        \tsTextMonospace{\tsBackslash{}includegraphics},
        e.g. \tsTextMonospace{width=380px}.                                                                  \\
        \hline
        % ----------------------------------------------------------------------
        \tsTextMonospace{\tsBackslash{}tsImageF\{filename\}\{description\}\newline\{options\}}             &
        Like \tsTextMonospace{\tsBackslash{}tsImage}, but with a frame.                                      \\
        \hline
        % ----------------------------------------------------------------------
        \tsTextMonospace{\tsBackslash{}tsImageO\{filename\}\{description\}\newline\{overlay\}\{options\}}  &
        Like \tsTextMonospace{\tsBackslash{}tsImage}, but with an overlaid text.                             \\
        \hline
        % ----------------------------------------------------------------------
        \tsTextMonospace{\tsBackslash{}tsImageOF\{filename\}\{description\}\newline\{overlay\}\{options\}} &
        Like \tsTextMonospace{\tsBackslash{}tsImageF}, but with an overlaid text.                            \\
        \hline
        % ----------------------------------------------------------------------
        \begin{BVerbatim}
\begin{tsLTItemize}
    \item Item I of enumeration list.
    \item Item B of enumeration list.
    \item Item 3 of enumeration list.
\end{tsLTItemize}
        \end{BVerbatim}
                                                                                                           &
        An enumeration list
        \begin{tsLTItemize}
            \item Item I of enumeration list.
            \item Item B of enumeration list.
            \item Item 3 of enumeration list.
        \end{tsLTItemize}
        \\
        \hline
        % ----------------------------------------------------------------------
        \tsTextMonospace{\tsBackslash{}tsNP\{\}}                                                           &
        This is used to start a new paragraph.                                                               \\
        \hline
        % ----------------------------------------------------------------------
        \tsTextMonospace{\tsBackslash{}tsTextBold\{text\}}                                                 &
        \tsTextBold{text}                                                                                    \\
        \hline
        % ----------------------------------------------------------------------
        \tsTextMonospace{\tsBackslash{}tsTextItalic\{text\}}                                               &
        \tsTextItalic{text}                                                                                  \\
        \hline
        % ----------------------------------------------------------------------
        \tsTextMonospace{\tsBackslash{}tsTextLower\{text\}}                                                &
        \tsTextLower{text}                                                                                   \\
        \hline
        % ----------------------------------------------------------------------
        \tsTextMonospace{\tsBackslash{}tsTextMonospace\{text\}}                                            &
        \tsTextMonospace{text}                                                                               \\
        \hline
        % ----------------------------------------------------------------------
        \tsTextMonospace{\tsBackslash{}tsTextQuoteF\{text\}}                                               &
        \tsTextQuoteF{text}                                                                                  \\
        \hline
        % ----------------------------------------------------------------------
        \tsTextMonospace{\tsBackslash{}tsTextQuoteG\{text\}}                                               &
        \tsTextQuoteG{text}                                                                                  \\
        \hline
        % ----------------------------------------------------------------------
        \tsTextMonospace{\tsBackslash{}tsTextStrikethrough\{text\}}                                        &
        \tsTextStrikethrough{text}                                                                           \\
        \hline
        % ----------------------------------------------------------------------
        \tsTextMonospace{\tsBackslash{}tsTextUnderline\{text\}}                                            &
        \tsTextUnderline{text}                                                                               \\
        \hline
        % ----------------------------------------------------------------------
        \tsTextMonospace{\tsBackslash{}tsTextUpper\{text\}}                                                &
        \tsTextUpper{text}                                                                                   \\
        \hline
        % ----------------------------------------------------------------------
        \tsCaptionLabelTable{Available commands}
    \end{longtable}
\end{footnotesize}

A look into the source code of this document shows the usage of the commands.

\newpage
\subsection{Examples to images}

\begin{figure}[H]
    \small
    \centering
    \begin{BVerbatim}
% Insert image TSTemplate-Logo.png
% Signature is 'Image without parameters'
% Optional parameters are empty
\tsImage{Images/TSTemplate-Logo.png}{Image without parameters}{}
    \end{BVerbatim}
    \tsCaptionLabelFigure{Code of image without parameters}
\end{figure}

\tsImage{Images/TSTemplate-Logo.png}{Image without parameters}{}

\begin{figure}[H]
    \small
    \centering
    \begin{BVerbatim}
% Insert image TSTemplate-Logo.png
% Signature is 'Image with parameter'
% Parameter: width=0.5\linewidth => scale the image to half page width
\tsImage{Images/TSTemplate-Logo.png}{Image with parameter}{width=0.5\linewidth}
    \end{BVerbatim}
    \tsCaptionLabelFigure{Code of image with parameter}
\end{figure}

\tsImage{Images/TSTemplate-Logo.png}{Image with parameter}{width=0.5\linewidth}

\begin{figure}[H]
    \small
    \centering
    \begin{BVerbatim}
% Insert image TSTemplate-Logo.png with frame
% Signature is 'Image without parameters, with frame'
% Optional parameters are empty
\tsImageF{Images/TSTemplate-Logo.png}{Image without parameters, with frame}{}
    \end{BVerbatim}
    \tsCaptionLabelFigure{Code of image without parameters, with frame}
\end{figure}

\tsImageF{Images/TSTemplate-Logo.png}{Image without parameters, with frame}{}

\begin{figure}[H]
    \small
    \centering
    \begin{BVerbatim}
% Insert image Garden.jpg with frame
% Signature is 'Image with overlay and parameters'
% Optional parameters are 'width=0.9\linewidth
\tsImageOF{./Images/Garden.jpg}{Image with frame, overlay
	and parameters}{\tsTextUpper{(C)} 2008, ThirtySomething}
{width=0.9\linewidth}
    \end{BVerbatim}
    \tsCaptionLabelFigure{Code image with frame, overlay and parameters}
\end{figure}

\tsImageOF{./Images/Garden.jpg}{Image with frame, overlay and parameters}{\copyright{} 2008, ThirtySomething}{width=0.9\linewidth}
