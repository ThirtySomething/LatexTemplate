%% -----------------------------------------------------------------------------
%% MIT License
%%
%% Copyright 2022 ThirtySomething
%%
%% Permission is hereby granted, free of charge, to any person obtaining a
%% copy of this software and associated documentation files (the "Software"),
%% to deal in the Software without restriction, including without limitation
%% the rights to use, copy, modify, merge, publish, distribute, sublicense,
%% and/or sell copies of the Software, and to permit persons to whom the
%% Software is furnished to do so, subject to the following conditions:
%%
%% The above copyright notice and this permission notice shall be included
%% in all copies or substantial portions of the Software.
%%
%% THE SOFTWARE IS PROVIDED "AS IS", WITHOUT WARRANTY OF ANY KIND, EXPRESS
%% OR IMPLIED, INCLUDING BUT NOT LIMITED TO THE WARRANTIES OF MERCHANTABILITY,
%% FITNESS FOR A PARTICULAR PURPOSE AND NONINFRINGEMENT. IN NO EVENT SHALL
%% THE AUTHORS OR COPYRIGHT HOLDERS BE LIABLE FOR ANY CLAIM, DAMAGES OR OTHER
%% LIABILITY, WHETHER IN AN ACTION OF CONTRACT, TORT OR OTHERWISE, ARISING
%% FROM, OUT OF OR IN CONNECTION WITH THE SOFTWARE OR THE USE OR OTHER
%% DEALINGS IN THE SOFTWARE.
%% -----------------------------------------------------------------------------

\section{Tables}

Font sizes play a role in the use of tables when it comes to distributing
the content appropriately among the columns. Here is a sample code how to
make a table.

\begin{figure}[H]
    \small
    \centering
    \begin{BVerbatim}
\begin{small}
    \renewcommand*{\arraystretch}{3}
    \begin{longtable}{ | p{0.43\linewidth} | p{0.5\linewidth} | }
        \hline
        \tsTextBold{Command}       & \tsTextBold{Sample} \\
        \hline
        \tsBackslash{}Huge\{text\} & \Huge{Huge}         \\
        \hline
        \tsCaptionLabelTable{Font sizes}
    \end{longtable}
\end{small}
    \end{BVerbatim}
    \tsCaptionLabelFigure{Tables}
\end{figure}

It is started with a block in one of the desired \nameref{sec:Font sizes}.
The first command within the font environment changes the spacing between the
text lines. If you do not do this, the table content and the table grid will
overlap or the spacing will be very tight. A good value is

\begin{figure}[H]
    \small
    \centering
    \begin{BVerbatim}
\renewcommand*{\arraystretch}{1.5}
    \end{BVerbatim}
    \tsCaptionLabelFigure{Row height in tables}
\end{figure}

The definition of the columns and their widths follows. The total width
available to a text in a line is \tsTextMonospace{\tsBackslash{}textwidth}. Due
to the lines of the table, not quite \tsTextMonospace{1.0\tsBackslash{}textwidth}
is available. So the sum of all individual column widths should not exceed
\tsTextMonospace{0.9\tsBackslash{}textwidth}.
\bigbreak

To slightly separate column headings from the column contents, the\linebreak
\tsTextMonospace{\tsBackslash{}tsTextBold\{column name\}} command can be used. Each
column value is separated from the next column by a \tsTextMonospace{\&}. The end
of the row is marked with a \tsTextMonospace{\tsBackslash{}\tsBackslash{}}.
\bigbreak

For horizontal lines, a \tsTextMonospace{\tsBackslash{}hline} is inserted.
\bigbreak

The data lines follow the same structure.
\bigbreak

To give the table a referencable name, add a
\tsTextMonospace{\tsBackslash{}tsCaptionLabelTable\{Font sizes\}} at the end.
Check the source of this document in the section \nameref{sec:Font sizes}
to see how it works.
\bigbreak

When using a \tsTextMonospace{verbatim} environment in a table, then the
column type of the table \tsTextBold{must be} of type \tsTextMonospace{P}!
Also important is that the column separator \tsTextMonospace{\&} must be on
the next line after the \tsTextMonospace{\tsBackslash{}end\{tsLTItemize\}}.
