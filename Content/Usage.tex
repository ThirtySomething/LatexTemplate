%% -----------------------------------------------------------------------------
%% MIT License
%%
%% Copyright 2022 ThirtySomething
%%
%% Permission is hereby granted, free of charge, to any person obtaining a
%% copy of this software and associated documentation files (the "Software"),
%% to deal in the Software without restriction, including without limitation
%% the rights to use, copy, modify, merge, publish, distribute, sublicense,
%% and/or sell copies of the Software, and to permit persons to whom the
%% Software is furnished to do so, subject to the following conditions:
%%
%% The above copyright notice and this permission notice shall be included
%% in all copies or substantial portions of the Software.
%%
%% THE SOFTWARE IS PROVIDED "AS IS", WITHOUT WARRANTY OF ANY KIND, EXPRESS
%% OR IMPLIED, INCLUDING BUT NOT LIMITED TO THE WARRANTIES OF MERCHANTABILITY,
%% FITNESS FOR A PARTICULAR PURPOSE AND NONINFRINGEMENT. IN NO EVENT SHALL
%% THE AUTHORS OR COPYRIGHT HOLDERS BE LIABLE FOR ANY CLAIM, DAMAGES OR OTHER
%% LIABILITY, WHETHER IN AN ACTION OF CONTRACT, TORT OR OTHERWISE, ARISING
%% FROM, OUT OF OR IN CONNECTION WITH THE SOFTWARE OR THE USE OR OTHER
%% DEALINGS IN THE SOFTWARE.
%% -----------------------------------------------------------------------------

\section{Usage}

As already described, the use of the template should be simple. So the user
copies the template, that is, the entire directory with all files and
subdirectories.
\bigbreak

Normally the user adjusts the metadata (file \tsTextItalic{TSTemplate-Meta.tex},
see \nameref{subsec:Metadata-I} and \nameref{subsec:Metadata-II}) and the
content in the file \tsTextItalic{TSTemplate-Content.tex} according to his
needs. Everything else is managed by the template.
\bigbreak

It is recommended to rename the main template file \tsTextItalic{TSTemplate.tex}
into a more more descriptive name.

\tsImageF{./Images/OtherPage.png}{Document page}{height=0.5\textheight}

\subsection{Metadata I}\label{subsec:Metadata-I}

For each document there are some metadata. This means the following definitions:

\begin{footnotesize}
    \renewcommand*{\arraystretch}{1.5}
    \begin{longtable}{ | p{0.43\linewidth} | p{0.5\linewidth} | }
        \hline
        \tsTextBold{Metadefinition}                         & \tsTextBold{Meaning}                                     \\
        \hline
        % ----------------------------------------------------------------------
        \tsTextMonospace{\tsBackslash{}tsAuthor\{\}}        & The author of the document.                              \\
        \hline
        % ----------------------------------------------------------------------
        \tsTextMonospace{\tsBackslash{}tsSubject\{\}}       & The subject of the document.                             \\
        \hline
        % ----------------------------------------------------------------------
        \tsTextMonospace{\tsBackslash{}tsTitle\{\}}         & The title of the document.                               \\
        \hline
        % ----------------------------------------------------------------------
        \tsTextMonospace{\tsBackslash{}tsURL\{\}}           & The url of the author.                                   \\
        \hline
        % ----------------------------------------------------------------------
        \tsTextMonospace{\tsBackslash{}tsVersionMajor\{\}}  & The major version of the document.\tsFootnoteDef{See
        \href{https://semver.org}{Semantic Versioning} for more details.}{semver}                                      \\
        \hline
        % ----------------------------------------------------------------------
        \tsTextMonospace{\tsBackslash{}tsVersionMinor\{\}}  & The minor version of the document.\tsFootnoteRef{semver} \\
        \hline
        % ----------------------------------------------------------------------
        \tsTextMonospace{\tsBackslash{}tsVersionPatch\{\}}  & The patch number of the document.\tsFootnoteRef{semver}  \\
        \hline
        % ----------------------------------------------------------------------
        \tsTextMonospace{\tsBackslash{}tsVersionString\{\}} & Complete version string.\tsFootnoteRef{semver}           \\
        \hline
        % ----------------------------------------------------------------------
        \tsCaptionLabelTable{Metadata I}
    \end{longtable}
\end{footnotesize}

These metadata are referenced in various places in the template. For example,
they are used in the title page of the document. The version number is based
on semantic versioning\tsFootnoteRef{semver}.

\tsImageF{./Images/TitlePage.png}{Title page}{height=0.5\textheight}

They are also used to set metadata in the generated PDF:

\tsImage{./Images/PDF-Properties.png}{PDF properties}{width=0.6\linewidth}

\subsection{Metadata II}\label{subsec:Metadata-II}

For further control, there are variables that can be used to enable/disable
which features are active for the document. These variables can be found
in the file \tsTextItalic{TSTemplate-Meta.tex}. There are:

\begin{footnotesize}
    \renewcommand*{\arraystretch}{1.5}
    \begin{longtable}{ | P{0.7\linewidth} | p{0.23\linewidth} | }
        \hline
        \tsTextBold{Metadefinition}
         & \tsTextBold{Beschreibung}                                           \\
        \hline
        % ----------------------------------------------------------------------
        \begin{BVerbatim}
% Definition of document type, either article or book
\newcommand{\tsDocumentType}[1]{article}
% \newcommand{\tsDocumentType}[1]{book}
        \end{BVerbatim}
         & Toggle between one page layout (article) and two page layout (book) \\
        \hline
        % ----------------------------------------------------------------------
        \begin{BVerbatim}
% Definition of document language
\newcommand{\tsDocumentLanguage}[1]{american}
% \newcommand{\tsDocumentLanguage}[1]{ngerman}
        \end{BVerbatim}
         & Setup of used language of document                                  \\
        \hline
        % ----------------------------------------------------------------------
        \begin{BVerbatim}
% Definition of page label
\newcommand{\tsDocumentPage}[1]{Page}
% \newcommand{\tsDocumentPage}[1]{Seite}
        \end{BVerbatim}
         & Setup of used label in front of the page number                     \\
        \hline
        % ----------------------------------------------------------------------
        \begin{BVerbatim}
% Switch to enable appendix
\newif\ifUseTsAppendix
% \UseTsAppendixtrue    % Disable appendix
\UseTsAppendixtrue      % Enable appendix
        \end{BVerbatim}
         & Toggle appendix                                                     \\
        \hline
        % ----------------------------------------------------------------------
        \begin{BVerbatim}
% Switch to enable bibliography and cites
\newif\ifUseTsCitation
% \UseTsCitationtrue    % Disable references
\UseTsCitationtrue      % Enable references
        \end{BVerbatim}
         & Toggle references                                                   \\
        \hline
        % ----------------------------------------------------------------------
        \begin{BVerbatim}
% Switch to enable glossary
\newif\ifUseTsGlossary
% \UseTsGlossarytrue    % Disable glossary
\UseTsGlossarytrue      % Enable glossary
        \end{BVerbatim}
         & Toggle glossary                                                     \\
        \hline
        % ----------------------------------------------------------------------
        \begin{BVerbatim}
% Switch to enable index
\newif\ifUseTsIndex
% \UseTsIndextrue       % Disable index
\UseTsIndextrue         % Enable index
        \end{BVerbatim}
         & Toggle index                                                        \\
        \hline
        % ----------------------------------------------------------------------
        \begin{BVerbatim}
% Switch to enable history
\newif\ifUseTsHistory
% \UseTsHistorytrue     % Disable history
\UseTsHistorytrue       % Enable history
        \end{BVerbatim}
         & Toggle history                                                      \\
        \hline
        % ----------------------------------------------------------------------
        \begin{BVerbatim}
% Switch to enable hyphenation
\newif\ifUseTsHyphenation
% \UseTsHyphenationtrue % Disable hyphenation
\UseTsHyphenationtrue   % Enable hyphenation
        \end{BVerbatim}
         & Toggle \newline hyphenation                                         \\
        \hline
        % ----------------------------------------------------------------------
        \begin{BVerbatim}
% Switch to enable list of figures
\newif\ifUseTsLOF
% \UseTsLOFtrue         % Disable list of figures
\UseTsLOFtrue           % Enable list of figures
        \end{BVerbatim}
         & Toggle list \newline of figures                                     \\
        \hline
        % ----------------------------------------------------------------------
        \begin{BVerbatim}
% Switch to enable logo
\newif\ifUseTsLogo
% \UseTsLogotrue        % Disable logo
\UseTsLogotrue          % Enable logo
        \end{BVerbatim}
         & Toggle logo                                                         \\
        \hline
        % ----------------------------------------------------------------------
        \begin{BVerbatim}
% Switch to enable list of tables
\newif\ifUseTsLOT
% \UseTsLOTtrue         % Disable list of tables
\UseTsLOTtrue           % Enable list of tables
        \end{BVerbatim}
         & Toggle list \newline of tables                                      \\
        \hline
        % ----------------------------------------------------------------------
        \begin{BVerbatim}
% Switch to disable indentation of paragraphs
\newif\ifUseTsParagraphNoIndentation
% \UseTsParagraphNoIndentationtrue  % Indentation yes
\UseTsParagraphNoIndentationtrue    % Indentation no
        \end{BVerbatim}
         & Toggle indentation \newline of paragraphs                           \\
        \hline
        % ----------------------------------------------------------------------
        \begin{BVerbatim}
% Switch to enable start of section on new page
\newif\ifUseTsSectionOnNewPage
% \UseTsSectionOnNewPagetrue % Section on same page
\UseTsSectionOnNewPagetrue   % Section on new page
        \end{BVerbatim}
         & Toggle start \newline of section                                    \\
        \hline
        % ----------------------------------------------------------------------
        \begin{BVerbatim}
% Switch to enable title page
\newif\ifUseTsTitlePage
% \UseTsTitlePagetrue   % Disable title page
\UseTsTitlePagetrue     % Enable title page
        \end{BVerbatim}
         & Toggle title page                                                   \\
        \hline
        % ----------------------------------------------------------------------
        \begin{BVerbatim}
% Switch to enable table of contents
\newif\ifUseTsTOC
% \UseTsTOCtrue         % Disable table of contents
\UseTsTOCtrue           % Enable table of contents
        \end{BVerbatim}
         & Toggle table \newline of contents                                   \\
        \hline
        % ----------------------------------------------------------------------
        \tsCaptionLabelTable{Metadata II}
    \end{longtable}
\end{footnotesize}

It should be noted that only one line has an influence on the option. The last
two lines in the option list show the disabled (with the \% sign at the
beginning of the line) and the enabled (without the \% sign) state.
\bigbreak

\tsTextBold{ATTENTION:} Turning options off and/or on will affect the steps
required for compilation. For example, if the glossary is disabled, no
\tsTextMonospace{makeglossary} may be used during compilation. Otherwise this will
result in errors.


\subsection{Metadata III}\label{subsec:Metadata-III}

Additional to the metadata of section \ref{subsec:Metadata-I} there exists
metadata to influence the filenames. Except for the
\tsTextItalic{TSTemplate-Meta.tex} file all other files are used indirect
by commands. So there are these commands available:

\begin{footnotesize}
    \renewcommand*{\arraystretch}{1.5}
    \begin{longtable}{ | p{0.43\linewidth} | p{0.5\linewidth} | }
        \hline
        \tsTextBold{Metadefinition}                              & \tsTextBold{Meaning}                  \\
        \hline
        % ----------------------------------------------------------------------
        \tsTextMonospace{\tsBackslash{}tsAppendixFile\{\}}       & Filename for appendices, \newline
        default \tsTextItalic{TSTemplate-Appendix.tex}                                                   \\
        \hline
        % ----------------------------------------------------------------------
        \tsTextMonospace{\tsBackslash{}tsCommandsFile\{\}}       & Filename for commands, \newline
        default \tsTextItalic{TSTemplate-Commands.tex}                                                   \\
        \hline
        % ----------------------------------------------------------------------
        \tsTextMonospace{\tsBackslash{}tsContentFile\{\}}        & Filename for content,\newline
        default \tsTextItalic{TSTemplate-Content.tex}                                                    \\
        \hline
        % ----------------------------------------------------------------------
        \tsTextMonospace{\tsBackslash{}tsCustomCommandsFile\{\}} & Filename for custom commands,\newline
        default \tsTextItalic{TSTemplate-CustomCommands.tex}                                             \\
        \hline
        % ----------------------------------------------------------------------
        \tsTextMonospace{\tsBackslash{}tsGlossaryFile\{\}}       & Filename for glossary,\newline
        default \tsTextItalic{TSTemplate-Glossary.tex}                                                   \\
        \hline
        % ----------------------------------------------------------------------
        \tsTextMonospace{\tsBackslash{}tsHistoryFile\{\}}        & Filename for history,\newline
        default \tsTextItalic{TSTemplate-History.tex}                                                    \\
        \hline
        % ----------------------------------------------------------------------
        \tsTextMonospace{\tsBackslash{}tsHyphenationFile\{\}}    & Filename for hyphenation,\newline
        default \tsTextItalic{TSTemplate-Hyphenation.tex}                                                \\
        \hline
        % ----------------------------------------------------------------------
        \tsTextMonospace{\tsBackslash{}tsLogoFile\{\}}           & Filename for logo,\newline
        default \tsTextItalic{Images/TSTemplate-Logo.png}                                                \\
        \hline
        % ----------------------------------------------------------------------
        \tsTextMonospace{\tsBackslash{}tsReferencesFile\{\}}     & Filename for references,\newline
        default \tsTextItalic{TSTemplate.bib}                                                            \\
        \hline
        % ----------------------------------------------------------------------
        \tsTextMonospace{\tsBackslash{}tsTitlePageFile\{\}}      & Filename for title page,\newline
        default \tsTextItalic{TSTemplate-TitlePage.tex}                                                  \\
        \hline
        % ----------------------------------------------------------------------
        \tsCaptionLabelTable{Metadata III}
    \end{longtable}
\end{footnotesize}

\subsection{Document content}

The document content is entered like a normal \LaTeX{} document. For this the
file \tsTextItalic{TSTemplate-Content.tex} is used, in which the user enters
the text. The basic conditions such as the inclusion of packages are regulated
in the file \tsTextItalic{TSTemplate.tex}. Features are enabled/disabled in the
file \tsTextItalic{TSTemplate-Meta.tex} -- see also \nameref{subsec:Metadata-II}.

\subsection{Glossary}

The user has the option of entering his own glossary. The entries for the
glossary are done in the file \tsTextItalic{TSTemplate-Glossary.tex}. \gls{TS}
is entered here as an example. This looks like this:

\begin{figure}[H]
    \scriptsize
    \centering
    \begin{BVerbatim}
\newglossaryentry{TS}{name={ThirtySomething},description={A glossary entry}}
    \end{BVerbatim}
    \tsCaptionLabelFigure{Definition of a glossary entry}
\end{figure}

The values have the following correspondences:

\begin{itemize}
    \item \tsTextBold{TS} -- This is the abbreviation used to reference the
          keyword in the document.
    \item \tsTextBold{ThirtySomething} -- This is the keyword listed in the
          glossary.
    \item \tsTextBold{A glossary entry} -- This is the description for the
          keyword listed in the glossary.
\end{itemize}

The glossary then lists the keyword, the description and the page number(s)
where the keyword was used everywhere. A prerequisite for this is, of course,
the use of the command \tsTextMonospace{\tsBackslash{}gls\{abbreviation\}} with
the existing abbreviation. In the above example, the usage looks like this:

\begin{figure}[H]
    \small
    \centering
    \begin{BVerbatim}
\gls{TS}
    \end{BVerbatim}
    \tsCaptionLabelFigure{Use of a glossary entry}
\end{figure}

In case a glossary is to be used, the document must be translated with
\tsTextMonospace{makeglossaries}.

\begin{figure}[H]
    \small
    \centering
    \begin{BVerbatim}
pdflatex
makeglossaries
pdflatex
pdflatex
    \end{BVerbatim}
    \tsCaptionLabelFigure{Translate with glossary}
\end{figure}

\subsection{Hyphenation}

In some circumstances, hyphenation is not performed for some words. This may
be because these words are very specific and \LaTeX{} does not know them and
therefore cannot hyphenate them. The result may be overlong lines, which are
clearly visible in most cases. In some cases this overlength is not
immediately obvious. For this you can temporarily add the line

\begin{figure}[H]
    \small
    \centering
    \begin{BVerbatim}
% To see the (page)frames used for hyphenation determination
% \usepackage{showframe}
    \end{BVerbatim}
    \tsCaptionLabelFigure{Preamble -- Package showframe}
\end{figure}

by removing the comment sign (\%). This will display frames in the document
indicating the text areas -- and you can clearly see the overlong lines. The
affected words at the end of the line can then be entered in the
\tsTextItalic{TSTemplate-Hyphenation.tex} file. An entry there follows the
following pattern:

\begin{figure}[H]
    \small
    \centering
    \begin{BVerbatim}
\hyphenation{Thir-ty-Some-thing}
    \end{BVerbatim}
    \tsCaptionLabelFigure{Exemplary content of TSHyphenation.tex}
\end{figure}

An entry corresponds to the word to be separated with a dash after each
syllable. If necessary in the different cases, then the words are separated
by a space. In addition, the following parameter can be changed in the document
\tsTextItalic{TSTemplate.tex}:

\begin{figure}[H]
    \small
    \centering
    \begin{BVerbatim}
% Setting tolerance for word wrap
% See also: https://texfaq.org/FAQ-overfull
\tolerance=3000
    \end{BVerbatim}
    \tsCaptionLabelFigure{Influencing the hyphenation}
\end{figure}

The permissible values are in the range from 0 to 10000.

\subsection{Bibliography references}\label{subsec:Bibliography references}

In case you want to have some bibliography references and want to cite, you
have to enable the citations in \tsTextItalic{TSTemplate-Meta.tex}:

\begin{figure}[H]
    \small
    \centering
    \begin{BVerbatim}
% Switch to enable bibliography and cites
\newif\ifUseTsCitation
\UseTsCitationtrue
    \end{BVerbatim}
    \tsCaptionLabelFigure{Enabled citations}
\end{figure}

Fill up your bibliography reference file:

\begin{figure}[H]
    \small
    \centering
    \begin{BVerbatim}
@misc{lfc,
    title        = {{Latex Font Catalogue}},
    author       = {},
    year         = {},
    note         = {Online checked on 16.06.2022},
    howpublished = {\url{https://tug.org/FontCatalogue/}}
}
    \end{BVerbatim}
    \tsCaptionLabelFigure{Reference definition}
\end{figure}

And to use it in the text.

\begin{figure}[H]
    \small
    \centering
    \begin{BVerbatim}
Fonts can be viewed at Latex Font Cataloge.\tsCitation{lfc}
    \end{BVerbatim}
    \tsCaptionLabelFigure{Reference usage}
\end{figure}

Fonts can be viewed at Latex Font Cataloge.\tsCitation{lfc}
