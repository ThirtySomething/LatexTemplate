%% -----------------------------------------------------------------------------
%% MIT License
%%
%% Copyright 2022 ThirtySomething
%%
%% Permission is hereby granted, free of charge, to any person obtaining a
%% copy of this software and associated documentation files (the "Software"),
%% to deal in the Software without restriction, including without limitation
%% the rights to use, copy, modify, merge, publish, distribute, sublicense,
%% and/or sell copies of the Software, and to permit persons to whom the
%% Software is furnished to do so, subject to the following conditions:
%%
%% The above copyright notice and this permission notice shall be included
%% in all copies or substantial portions of the Software.
%%
%% THE SOFTWARE IS PROVIDED "AS IS", WITHOUT WARRANTY OF ANY KIND, EXPRESS
%% OR IMPLIED, INCLUDING BUT NOT LIMITED TO THE WARRANTIES OF MERCHANTABILITY,
%% FITNESS FOR A PARTICULAR PURPOSE AND NONINFRINGEMENT. IN NO EVENT SHALL
%% THE AUTHORS OR COPYRIGHT HOLDERS BE LIABLE FOR ANY CLAIM, DAMAGES OR OTHER
%% LIABILITY, WHETHER IN AN ACTION OF CONTRACT, TORT OR OTHERWISE, ARISING
%% FROM, OUT OF OR IN CONNECTION WITH THE SOFTWARE OR THE USE OR OTHER
%% DEALINGS IN THE SOFTWARE.
%% -----------------------------------------------------------------------------

%% -----------------------------------------------------------------------------
%% Common command definition of Latex template
%% -----------------------------------------------------------------------------

%% -----------------------------------------------------------------------------
%% Definition of internal commands
%% -----------------------------------------------------------------------------
% Enable underscore as legal character
% https://tex.stackexchange.com/questions/359787/is-it-safe-to-set-underscore-to-a-non-active-character
\catcode`_=12
% Define common date format: DD.MM.YYYY
\newdateformat{datedot}{\twodigit{\THEDAY}.\twodigit{\THEMONTH}.\THEYEAR}
% New columntype for longtables used for itemize environment
\newcolumntype{P}[1]{>{\endgraf\vspace*{-\baselineskip}}p{#1}}
% For books insert an empty page after title page
\newcommand{\tsEmptyPage}[1]
{
    \ifthenelse{\equal{#1}{book}}
    {
        \newpage\null\thispagestyle{empty}\newpage
    }
    {
    }
}
% Define new command for header/footer of article
\newcommand{\tsFancyArticle}[1]
{
    \fancyhead[L]{\scriptsize\tsTitle{}}
    \ifUseTsLogo
        \fancyhead[R]{\includegraphics[height=0.7\headheight]{\tsLogoFile{}}}
    \else
        \fancyhead[R]{\scriptsize\tsURL{}}
    \fi
    \fancyfoot[L]{\scriptsize\tsToday{}}
    \fancyfoot[R]{\scriptsize\tsDocumentPage{} \thepage}
}
% Define new command for header/footer of book
\newcommand{\tsFancyBook}[1]
{
    \fancyhead[LO]{\scriptsize\tsTitle{}}
    \fancyhead[RE]{\scriptsize\tsTitle{}}
    \ifUseTsLogo
        \fancyhead[RO]{\includegraphics[height=0.7\headheight]{\tsLogoFile{}}}
        \fancyhead[LE]{\includegraphics[height=0.7\headheight]{\tsLogoFile{}}}
    \else
        \fancyhead[RO]{\scriptsize\tsURL{}}
        \fancyhead[LE]{\scriptsize\tsURL{}}
    \fi
    \fancyfoot[LE]{\scriptsize\tsDocumentPage{} \thepage}
    \fancyfoot[LO]{\scriptsize\tsToday{}}
    \fancyfoot[RE]{\scriptsize\tsToday{}}
    \fancyfoot[RO]{\scriptsize\tsDocumentPage{} \thepage}
}
% Define command to choose header/footer depending on argument
\newcommand{\tsFancyData}[1]{\ifthenelse{\equal{#1}{article}}{\tsFancyArticle{}}{\tsFancyBook{}}}
% Command for enabling dottet lines in table of content only for books
% https://tex.stackexchange.com/questions/53898/how-to-get-lines-with-dots-in-the-table-of-contents-for-sections
\newcommand{\tsTOCDottedLine}[1]
{
    \ifthenelse{\equal{#1}{book}}
    {
        \renewcommand{\cftpartleader}{\cftdotfill{\cftdotsep}}
        \renewcommand{\cftchapleader}{\cftdotfill{\cftdotsep}}
        \renewcommand{\cftsecleader}{\cftdotfill{\cftdotsep}}
    }
    {
    }
}
% Enable numeric footnote inside tables
\renewcommand{\thempfootnote}{\arabic{mpfootnote}}

%% -----------------------------------------------------------------------------
%% Definition of public commands
%% -----------------------------------------------------------------------------

% Definition of a file as appendix
\newcommand{\tsAppendixIncludeFile}[1]
{
    \tsAppendixSection{#1}
    \begin{scriptsize}
        \VerbatimInput{#1}
    \end{scriptsize}
}
% Definition for convenience usage of appendix
\newcommand{\tsAppendixSection}[1]
{
    \newpage
    \section{#1}\label{appendix:#1}
}
% Definition of arrow right
\newcommand{\tsArrowRight}[1]{$\rightarrow$}
% Definition of arrow down
\newcommand{\tsArrowDown}[1]{$\downarrow$}
% Definition of arrow left
\newcommand{\tsArrowLeft}[1]{$\leftarrow$}
% Definition of arrow up
\newcommand{\tsArrowUp}[1]{$\uparrow$}
% Definition of double arrow right
\newcommand{\tsArrowRightDouble}[1]{$\Rightarrow$}
% Definition of double arrow down
\newcommand{\tsArrowDownDouble}[1]{$\Downarrow$}
% Definition of double arrow left
\newcommand{\tsArrowLeftDouble}[1]{$\Leftarrow$}
% Definition of double arrow up
\newcommand{\tsArrowUpDouble}[1]{$\Uparrow$}
% Define command backslash
\newcommand{\tsBackslash}[1]{\textbackslash}
% Define a bullet
\newcommand{\tsBullet}[1]{\textbullet{}}
% Define command for caption including label for figures
\newcommand{\tsCaptionLabelFigure}[1]{\caption{#1}\label{fig:#1}}
% Define command for caption including label for formulas
\newcommand{\tsCaptionLabelFormula}[1]{\caption{#1}\label{for:#1}}
% Define command for caption including label for tables
\newcommand{\tsCaptionLabelTable}[1]{\caption{#1}\label{tab:#1}}
% Define convenience command for citations
\newcommand{\tsCitation}[1]{~\cite{#1}}
% Define convenience command for degree
\newcommand{\tsDegree}[1]{#1$^\circ$}
% Define convenience command for diameter
\newcommand{\tsDiameter}[1]{\diameter}
% Define command of a frac like 1/2
\newcommand{\tsFrac}[2]{\tsTextUpper{#1}\kern-.06em/\kern-0.06em\tsTextLower{#2}}
% Define a referencable footnote
\newcommand{\tsFootnoteDef}[2]{\footnote{\label{#2}#1}}
% Define a referencable footnote
\newcommand{\tsFootnoteRef}[1]{\tsTextUpper{\ref{#1}}}
% Define command to include an image without a frame
\newcommand{\tsImage}[3]
{
    \begin{figure}[H]
        \centering
        \includegraphics[#3]{#1}
        \tsCaptionLabelFigure{#2}
    \end{figure}
}
% Define command to include an image with a frame
\newcommand{\tsImageF}[3]
{
    \begin{figure}[H]
        \centering
        \fbox{\includegraphics[#3]{#1}}
        \tsCaptionLabelFigure{#2}
    \end{figure}
}
% Define command to include an image without a frame but an overlaid text
\newcommand{\tsImageO}[4]
{
    \begin{figure}[H]
        \centering
        \begin{overpic}[permil,#4]{#1}
            \put(10,10){\textcolor{white}{\scriptsize\contour{black}{#3}}}
        \end{overpic}
        \tsCaptionLabelFigure{#2}
    \end{figure}
}
% Define command to include an image with a frame and an overlaid text
\newcommand{\tsImageOF}[4]
{
    \begin{figure}[H]
        \centering
        \fbox{
            \begin{overpic}[permil,#4]{#1}
                \put(10,10){\textcolor{white}{\scriptsize\contour{black}{#3}}}
            \end{overpic}
        }
        \tsCaptionLabelFigure{#2}
    \end{figure}
}
% To start a new paragraph
\newcommand{\tsNP}[1]{\bigbreak}
% Define command for text in bold
\newcommand{\tsTextBold}[1]{\textbf{#1}}
% Define command for text in italic
\newcommand{\tsTextItalic}[1]{\emph{#1}}
% Define command for text in subscript
\newcommand{\tsTextLower}[1]{\textsubscript{#1}}
% Define command for text in monospace
\newcommand{\tsTextMonospace}[1]{\texttt{#1}}
% Define command for text in french quotes
\newcommand{\tsTextQuoteF}[1]{\guillemotleft{}#1\guillemotright{}}
% Define command for text in german quotes
\newcommand{\tsTextQuoteG}[1]{\glqq{}#1\grqq{}}
% Define command for text strikethrough
\newcommand{\tsTextStrikethrough}[1]{\sout{#1}}
% Define command for underlined text
\newcommand{\tsTextUnderline}[1]{\underline{#1}}
% Define command for text in superscript
\newcommand{\tsTextUpper}[1]{\textsuperscript{#1}}
% Definition of today in new date format
\newcommand{\tsToday}[1]{\datedot\today}
% Define new environment for itemize used in longtables
\newenvironment{tsLTItemize}
{
    \begin{itemize}[noitemsep,leftmargin=*,topsep=0pt,partopsep=0pt]}{\vspace*{-\baselineskip}
    \end{itemize}
}
