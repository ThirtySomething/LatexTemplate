%% -----------------------------------------------------------------------------
%% MIT License
%%
%% Copyright 2022 ThirtySomething
%%
%% Permission is hereby granted, free of charge, to any person obtaining a
%% copy of this software and associated documentation files (the "Software"),
%% to deal in the Software without restriction, including without limitation
%% the rights to use, copy, modify, merge, publish, distribute, sublicense,
%% and/or sell copies of the Software, and to permit persons to whom the
%% Software is furnished to do so, subject to the following conditions:
%%
%% The above copyright notice and this permission notice shall be included
%% in all copies or substantial portions of the Software.
%%
%% THE SOFTWARE IS PROVIDED "AS IS", WITHOUT WARRANTY OF ANY KIND, EXPRESS
%% OR IMPLIED, INCLUDING BUT NOT LIMITED TO THE WARRANTIES OF MERCHANTABILITY,
%% FITNESS FOR A PARTICULAR PURPOSE AND NONINFRINGEMENT. IN NO EVENT SHALL
%% THE AUTHORS OR COPYRIGHT HOLDERS BE LIABLE FOR ANY CLAIM, DAMAGES OR OTHER
%% LIABILITY, WHETHER IN AN ACTION OF CONTRACT, TORT OR OTHERWISE, ARISING
%% FROM, OUT OF OR IN CONNECTION WITH THE SOFTWARE OR THE USE OR OTHER
%% DEALINGS IN THE SOFTWARE.
%% -----------------------------------------------------------------------------

%% -----------------------------------------------------------------------------
%% Common command definition of Latex template
%% -----------------------------------------------------------------------------

% Enable underscore as legal character
% https://tex.stackexchange.com/questions/359787/is-it-safe-to-set-underscore-to-a-non-active-character
\catcode`_=12
% New columntype for longtables used for itemize environment
\newcolumntype{P}[1]{>{\endgraf\vspace*{-\baselineskip}}p{#1}}
% Definition for convenience usage of appendix
\newcommand{\tsAppendixSection}[1]{\section{#1}\label{appendix:#1}}
% Definition of arrow right
\newcommand{\tsArrowRight}[1]{$\rightarrow$}
% Definition of arrow down
\newcommand{\tsArrowDown}[1]{$\downarrow$}
% Definition of arrow left
\newcommand{\tsArrowLeft}[1]{$\leftarrow$}
% Definition of arrow up
\newcommand{\tsArrowUp}[1]{$\uparrow$}
% Definition of double arrow right
\newcommand{\tsArrowRightDouble}[1]{$\Rightarrow$}
% Definition of double arrow down
\newcommand{\tsArrowDownDouble}[1]{$\Downarrow$}
% Definition of double arrow left
\newcommand{\tsArrowLeftDouble}[1]{$\Leftarrow$}
% Definition of double arrow up
\newcommand{\tsArrowUpDouble}[1]{$\Uparrow$}
% Define command backslash
\newcommand{\tsBackslash}[1]{\textbackslash}
% Define a bullet
\newcommand{\tsBullet}[1]{\textbullet{}}
% Define command for caption including label for figures
\newcommand{\tsCaptionLabelFigure}[1]{\caption{#1}\label{fig:#1}}
% Define command for caption including label for formulas
\newcommand{\tsCaptionLabelFormula}[1]{\caption{#1}\label{for:#1}}
% Define command for caption including label for tables
\newcommand{\tsCaptionLabelTable}[1]{\caption{#1}\label{tab:#1}}
% Define common date format: DD.MM.YYYY
\newdateformat{datedot}{\twodigit{\THEDAY}.\twodigit{\THEMONTH}.\THEYEAR}
% Definition of today in new date format
\newcommand{\tsToday}[1]{\datedot\today}
% Define convenience command for degree
\newcommand{\tsDegree}[1]{#1$^\circ$}
% Define command for bold text
\newcommand{\tsFontBold}[1]{\textbf{#1}}
% Define command for verbatim text
\newcommand{\tsFontCode}[1]{\small \texttt{#1}}
% Define command for italic text
% \newcommand{\tsFontItalic}[1]{\textit{#1}}
% See also https://de.overleaf.com/learn/latex/Bold%2C_italics_and_underlining
\newcommand{\tsFontItalic}[1]{\emph{#1}}
% Define command for underlined text
\newcommand{\tsFontUnderline}[1]{\underline{#1}}
% Define a referencable footnote
\newcommand{\tsFootnoteDef}[2]{\footnote{\label{#2}#1}}
% Define a referencable footnote
\newcommand{\tsFootnoteRef}[1]{\tsTextUpper{\ref{#1}}}
% Define command to include an image without a frame
\newcommand{\tsImage}[3]{\begin{figure}[H]\centering\includegraphics[#3]{#1}\tsCaptionLabelFigure{#2}\end{figure}}
% Define command to include an image with a frame
\newcommand{\tsImageF}[3]{\begin{figure}[H]\centering\fbox{\includegraphics[#3]{#1}}\tsCaptionLabelFigure{#2}\end{figure}}
% Define command for quoting text on french way
\newcommand{\tsTextQuoteF}[1]{\guillemotleft{}#1\guillemotright{}}
% Define command for quoting text on german way
\newcommand{\tsTextQuoteG}[1]{\glqq{}#1\grqq{}}
% Define command for text in subscript
\newcommand{\tsTextLower}[1]{\textsubscript{#1}}
% Define command for text in superscript
\newcommand{\tsTextUpper}[1]{\textsuperscript{#1}}
% Define new environment for itemize used in longtables
\newenvironment{tsLTItemize}{\begin{itemize}[noitemsep,leftmargin=*,topsep=0pt,partopsep=0pt]}{\vspace*{-\baselineskip}\end{itemize}}
