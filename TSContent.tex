%% -----------------------------------------------------------------------------
%% MIT License
%%
%% Copyright 2022 ThirtySomething
%%
%% Permission is hereby granted, free of charge, to any person obtaining a
%% copy of this software and associated documentation files (the "Software"),
%% to deal in the Software without restriction, including without limitation
%% the rights to use, copy, modify, merge, publish, distribute, sublicense,
%% and/or sell copies of the Software, and to permit persons to whom the
%% Software is furnished to do so, subject to the following conditions:
%%
%% The above copyright notice and this permission notice shall be included
%% in all copies or substantial portions of the Software.
%%
%% THE SOFTWARE IS PROVIDED "AS IS", WITHOUT WARRANTY OF ANY KIND, EXPRESS
%% OR IMPLIED, INCLUDING BUT NOT LIMITED TO THE WARRANTIES OF MERCHANTABILITY,
%% FITNESS FOR A PARTICULAR PURPOSE AND NONINFRINGEMENT. IN NO EVENT SHALL
%% THE AUTHORS OR COPYRIGHT HOLDERS BE LIABLE FOR ANY CLAIM, DAMAGES OR OTHER
%% LIABILITY, WHETHER IN AN ACTION OF CONTRACT, TORT OR OTHERWISE, ARISING
%% FROM, OUT OF OR IN CONNECTION WITH THE SOFTWARE OR THE USE OR OTHER
%% DEALINGS IN THE SOFTWARE.
%% -----------------------------------------------------------------------------

%% -----------------------------------------------------------------------------
%% Content definition of Latex template
%% -----------------------------------------------------------------------------

\section{License information}

This template is published under the \href{https://opensource.org/licenses/mit-license.php}{MIT license}:

\begin{figure}[H]
    \scriptsize
    \VerbatimInput{LICENSE}
    \tsCaptionLabelFigure{MIT License}
\end{figure}

\section{Introduction}

This is a template for use with \LaTeX{}. An attempt has been made to make
the usage of this template as simple as possible for the user.

\subsection{Nameing conventions}

For this document some naming conventions are used.

\begin{itemize}
    \item Text in \tsTextMonospace{monospace} is either a command or
          \LaTeX{} source code.
    \item Text in \tsTextItalic{italic} are usually filenames.
\end{itemize}

\section{Template}
\label{sec:Template}

The template is divided into several files, which are listed here below:

\begin{small}
    \renewcommand*{\arraystretch}{1.5}
    \begin{longtable}{ | p{0.45\textwidth} | p{0.5\textwidth} | }
        \hline
        \tsTextBold{File}                   & \tsTextBold{Meaning}            \\
        \hline
        % ----------------------------------------------------------------------
        \tsTextItalic{LICENSE}              & MIT license file.               \\
        \hline
        % ----------------------------------------------------------------------
        \tsTextItalic{README.md}            & Information about the template. \\
        \hline
        % ----------------------------------------------------------------------
        \tsTextItalic{TSAppendix.tex}       & The appendices of the document. \\
        \hline
        % ----------------------------------------------------------------------
        \tsTextItalic{TSCommands.tex}       & The common predefined commands. \\
        \hline
        % ----------------------------------------------------------------------
        \tsTextItalic{TSContent.tex}        & The content of the document.    \\
        \hline
        % ----------------------------------------------------------------------
        \tsTextItalic{TSCustomCommands.tex} & The document specific commands. \\
        \hline
        % ----------------------------------------------------------------------
        \tsTextItalic{TSGlossary.tex}       & The glossary of the document.   \\
        \hline
        % ----------------------------------------------------------------------
        \tsTextItalic{TSHistory.tex}        & Document history.               \\
        \hline
        % ----------------------------------------------------------------------
        \tsTextItalic{TSHyphenation.tex}    & Document-specific hyphenation.  \\
        \hline
        % ----------------------------------------------------------------------
        \tsTextItalic{TSTemplate-Meta.tex}  & The metadata of the document.   \\
        \hline
        % ----------------------------------------------------------------------
        \tsTextItalic{TSTemplate.pdf}       & The result as a PDF file.       \\
        \hline
        % ----------------------------------------------------------------------
        \tsTextItalic{TSTemplate.tex}       & The envelope document.          \\
        \hline
        % ----------------------------------------------------------------------
        \tsTextItalic{TSTitlePage.tex}      & The title page of the document. \\
        \hline
        % ----------------------------------------------------------------------
        \tsTextItalic{Images/TSLogo.png}    & The logo used.                  \\
        \hline
        % ----------------------------------------------------------------------
        \tsCaptionLabelTable{The files of the template}
    \end{longtable}
\end{small}

Some commands and formatting have been predefined for the template. They all
have in common that they start with a \tsTextMonospace{\tsBackslash{}ts\ldots{}}.
This serves to distinguish them from the commands otherwise used in \LaTeX{}.
The available commands are explained in the \nameref{sec:Commands} chapter.
\bigbreak

\tsTextBold{NOTE:} The file containing the meta data is depending on the name
of the envelope document \tsTextItalic{TSTemplate.tex}. The rule is just simple:
The name will be \tsTextItalic{TSTemplate} followed by \tsTextItalic{-Meta.tex}.
If you name your document \tsTextItalic{MyDocument.tex} then the filename of the
metadata file will be \tsTextItalic{MyDocument-Meta.tex}.

\section{Usage}

As already described, the use of the template should be simple. So the user
copies the template, that is, the entire directory with all files and
subdirectories.
\bigbreak

Normally the user adjusts the metadata (file \tsTextItalic{TSTemplate-Meta.tex},
see \nameref{subsec:Metadata-I} and \nameref{subsec:Metadata-II}) and the
content in the file \tsTextItalic{TSContent.tex} according to his needs.
Everything else is managed by the template.
\bigbreak

It is recommended to rename the main template file \tsTextItalic{TSTemplate.tex}
into a more more descriptive name.

\tsImageF{./Images/OtherPage.png}{Document page}{height=0.5\textheight}

\subsection{Metadata I}\label{subsec:Metadata-I}

For each document there are some metadata. This means the following definitions:

\begin{footnotesize}
    \renewcommand*{\arraystretch}{1.5}
    \begin{longtable}{ | p{0.45\textwidth} | p{0.5\textwidth} | }
        \hline
        \tsTextBold{Metadefinition}                         & \tsTextBold{Meaning}                                     \\
        \hline
        % ----------------------------------------------------------------------
        \tsTextMonospace{\tsBackslash{}tsAuthor\{\}}        & The author of the document.                              \\
        \hline
        % ----------------------------------------------------------------------
        \tsTextMonospace{\tsBackslash{}tsSubject\{\}}       & The subject of the document.                             \\
        \hline
        % ----------------------------------------------------------------------
        \tsTextMonospace{\tsBackslash{}tsTitle\{\}}         & The title of the document.                               \\
        \hline
        % ----------------------------------------------------------------------
        \tsTextMonospace{\tsBackslash{}tsURL\{\}}           & The url of the author.                                   \\
        \hline
        % ----------------------------------------------------------------------
        \tsTextMonospace{\tsBackslash{}tsVersionMajor\{\}}  & The major version of the document.\tsFootnoteDef{See
        \href{https://semver.org}{Semantic Versioning} for more details.}{semver}                                      \\
        \hline
        % ----------------------------------------------------------------------
        \tsTextMonospace{\tsBackslash{}tsVersionMinor\{\}}  & The minor version of the document.\tsFootnoteRef{semver} \\
        \hline
        % ----------------------------------------------------------------------
        \tsTextMonospace{\tsBackslash{}tsVersionPatch\{\}}  & The patch number of the document.\tsFootnoteRef{semver}  \\
        \hline
        % ----------------------------------------------------------------------
        \tsTextMonospace{\tsBackslash{}tsVersionString\{\}} & Complete version string.\tsFootnoteRef{semver}           \\
        \hline
        % ----------------------------------------------------------------------
        \tsCaptionLabelTable{Metadata I}
    \end{longtable}
\end{footnotesize}

These metadata are referenced in various places in the template. For example,
they are used in the title page of the document. The version number is based
on semantic versioning\tsFootnoteRef{semver}.

\tsImageF{./Images/TitlePage.png}{Title page}{height=0.5\textheight}

They are also used to set metadata in the generated PDF:

\tsImage{./Images/PDF-Properties.png}{PDF properties}{width=0.6\textwidth}

\subsection{Metadata II}\label{subsec:Metadata-II}

For further control, there are variables that can be used to enable/disable
which features are active for the document. These variables can be found
in the file \tsTextItalic{TSTemplate-Meta.tex}. There are:

\begin{footnotesize}
    \renewcommand*{\arraystretch}{1.5}
    \begin{longtable}{ | P{0.7\textwidth} | p{0.2\textwidth} | }
        \hline
        \tsTextBold{Metadefinition}
         & \tsTextBold{Beschreibung}         \\
        \hline
        % ----------------------------------------------------------------------
        \begin{BVerbatim}
% Switch to enable appendix
\newif\ifUseTsAppendix
% \UseTsAppendixtrue    % Disable appendix
\UseTsAppendixtrue      % Enable appendix
        \end{BVerbatim}
         & Toggle appendix                   \\
        \hline
        % ----------------------------------------------------------------------
        \begin{BVerbatim}
% Switch to enable glossary
\newif\ifUseTsGlossary
% \UseTsGlossarytrue    % Disable glossary
\UseTsGlossarytrue      % Enable glossary
        \end{BVerbatim}
         & Toggle glossary                   \\
        \hline
        % ----------------------------------------------------------------------
        \begin{BVerbatim}
% Switch to enable history
\newif\ifUseTsHistory
% \UseTsHistorytrue     % Disable history
\UseTsHistorytrue       % Enable history
        \end{BVerbatim}
         & Toggle history                    \\
        \hline
        % ----------------------------------------------------------------------
        \begin{BVerbatim}
% Switch to enable hyphenation
\newif\ifUseTsHyphenation
% \UseTsHyphenationtrue % Disable hyphenation
\UseTsHyphenationtrue   % Enable hyphenation
        \end{BVerbatim}
         & Toggle \newline  hyphenation      \\
        \hline
        % ----------------------------------------------------------------------
        \begin{BVerbatim}
% Switch to enable list of figures
\newif\ifUseTsLOF
% \UseTsLOFtrue         % Disable list of figures
\UseTsLOFtrue           % Enable list of figures
        \end{BVerbatim}
         & Toggle list \newline of figures   \\
        \hline
        % ----------------------------------------------------------------------
        \begin{BVerbatim}
% Switch to enable logo
\newif\ifUseTsLogo
% \UseTsLogotrue        % Disable logo
\UseTsLogotrue          % Enable logo
        \end{BVerbatim}
         & Toggle logo                       \\
        \hline
        % ----------------------------------------------------------------------
        \begin{BVerbatim}
% Switch to enable list of tables
\newif\ifUseTsLOT
% \UseTsLOTtrue         % Disable list of tables
\UseTsLOTtrue           % Enable list of tables
        \end{BVerbatim}
         & Toggle list \newline of tables    \\
        \hline
        % ----------------------------------------------------------------------
        \begin{BVerbatim}
% Switch to enable start of section on new page
\newif\ifUseTsSectionOnNewPage
% \UseTsSectionOnNewPagetrue % Section on same page
\UseTsSectionOnNewPagetrue   % Section on new page
        \end{BVerbatim}
         & Toggle start \newline of section  \\
        \hline
        % ----------------------------------------------------------------------
        \begin{BVerbatim}
% Switch to enable title page
\newif\ifUseTsTitlePage
% \UseTsTitlePagetrue   % Disable title page
\UseTsTitlePagetrue     % Enable title page
        \end{BVerbatim}
         & Toggle title page                 \\
        \hline
        % ----------------------------------------------------------------------
        \begin{BVerbatim}
% Switch to enable table of contents
\newif\ifUseTsTOC
% \UseTsTOCtrue         % Disable table of contents
\UseTsTOCtrue           % Enable table of contents
        \end{BVerbatim}
         & Toggle table \newline of contents \\
        \hline
        % ----------------------------------------------------------------------
        \tsCaptionLabelTable{Metadata II}
    \end{longtable}
\end{footnotesize}

It should be noted that only one line has an influence on the option. The last
two lines in the option list show the disabled (with the \% sign at the
beginning of the line) and the enabled (without the \% sign) state.
\bigbreak

\tsTextBold{ATTENTION:} Turning options off and/or on will affect the steps
required for compilation. For example, if the glossary is disabled, no
\tsTextMonospace{makeglossary} may be used during compilation. Otherwise this will
result in errors.

\subsection{Metadata III}\label{subsec:Metadata-III}

Additional to the metadata of section \ref{subsec:Metadata-I} there exists
metadata to influence the filenames. Except for the
\tsTextItalic{TSTemplate-Meta.tex} file all other files are used indirect
by commands. So there are these commands available:

\begin{footnotesize}
    \renewcommand*{\arraystretch}{1.5}
    \begin{longtable}{ | p{0.45\textwidth} | p{0.5\textwidth} | }
        \hline
        \tsTextBold{Metadefinition}                              & \tsTextBold{Meaning}                  \\
        \hline
        % ----------------------------------------------------------------------
        \tsTextMonospace{\tsBackslash{}tsAppendixFile\{\}}       & Filename for appendices, \newline
        default \tsTextItalic{TSAppendix.tex}                                                         \\
        \hline
        % ----------------------------------------------------------------------
        \tsTextMonospace{\tsBackslash{}tsCommandsFile\{\}}       & Filename for commands, \newline
        default \tsTextItalic{TSCommands.tex}                                                         \\
        \hline
        % ----------------------------------------------------------------------
        \tsTextMonospace{\tsBackslash{}tsContentFile\{\}}        & Filename for content,\newline
        default \tsTextItalic{TSContent.tex}                                                          \\
        \hline
        % ----------------------------------------------------------------------
        \tsTextMonospace{\tsBackslash{}tsCustomCommandsFile\{\}} & Filename for custom commands,\newline
        default \tsTextItalic{TSCustomCommands.tex}                                                   \\
        \hline
        % ----------------------------------------------------------------------
        \tsTextMonospace{\tsBackslash{}tsGlossaryFile\{\}}       & Filename for glossary,\newline
        default \tsTextItalic{TSGlossary.tex}                                                         \\
        \hline
        % ----------------------------------------------------------------------
        \tsTextMonospace{\tsBackslash{}tsHistoryFile\{\}}        & Filename for history,\newline
        default \tsTextItalic{TSHistory.tex}                                                          \\
        \hline
        % ----------------------------------------------------------------------
        \tsTextMonospace{\tsBackslash{}tsHyphenationFile\{\}}    & Filename for hyphenation,\newline
        default \tsTextItalic{TSHyphenation.tex}                                                      \\
        \hline
        % ----------------------------------------------------------------------
        \tsTextMonospace{\tsBackslash{}tsLogoFile\{\}}           & Filename for logo,\newline
        default \tsTextItalic{./Images/TSLogo.png}                                                    \\
        \hline
        % ----------------------------------------------------------------------
        \tsTextMonospace{\tsBackslash{}tsTitlePageFile\{\}}      & Filename for title page,\newline
        default \tsTextItalic{TSTitlePage.tex}                                                        \\
        \hline
        % ----------------------------------------------------------------------
        \tsCaptionLabelTable{Metadata III}
    \end{longtable}
\end{footnotesize}

\subsection{Document content}

The document content is entered like a normal \LaTeX{} document. For this the
file \tsTextItalic{TSContent.tex} is used, in which the user enters the text.
The basic conditions such as the inclusion of packages are regulated in the file
\tsTextItalic{TSTemplate.tex}. Features are enabled/disabled in the file
\tsTextItalic{TSTemplate-Meta.tex} -- see also \nameref{subsec:Metadata-II}.

\subsection{Glossary}

The user has the option of entering his own glossary. The entries for the
glossary are done in the file \tsTextItalic{TSGlossary.tex}. \gls{TS}
is entered here as an example. This looks like this:

\begin{figure}[H]
    \scriptsize
    \centering
    \begin{BVerbatim}
\newglossaryentry{TS}{name={ThirtySomething},description={A glossary entry}}
    \end{BVerbatim}
    \tsCaptionLabelFigure{Definition of a glossary entry}
\end{figure}

The values have the following correspondences:

\begin{itemize}
    \item \tsTextBold{TS} -- This is the abbreviation used to reference the
          keyword in the document.
    \item \tsTextBold{ThirtySomething} -- This is the keyword listed in the
          glossary.
    \item \tsTextBold{A glossary entry} -- This is the description for the
          keyword listed in the glossary.
\end{itemize}

The glossary then lists the keyword, the description and the page number(s)
where the keyword was used everywhere. A prerequisite for this is, of course,
the use of the command \tsTextMonospace{\tsBackslash{}gls\{abbreviation\}} with
the existing abbreviation. In the above example, the usage looks like this:

\begin{figure}[H]
    \small
    \centering
    \begin{BVerbatim}
\gls{TS}
    \end{BVerbatim}
    \tsCaptionLabelFigure{Use of a glossary entry}
\end{figure}

In case a glossary is to be used, the document must be translated with
\tsTextMonospace{makeglossaries}.

\begin{figure}[H]
    \small
    \centering
    \begin{BVerbatim}
pdflatex
makeglossaries
pdflatex
pdflatex
    \end{BVerbatim}
    \tsCaptionLabelFigure{Translate with glossary}
\end{figure}

\subsection{Hyphenation}

In some circumstances, hyphenation is not performed for some words. This may
be because these words are very specific and \LaTeX{} does not know them and
therefore cannot hyphenate them. The result may be overlong lines, which are
clearly visible in most cases. In some cases this overlength is not
immediately obvious. For this you can temporarily add the line

\begin{figure}[H]
    \small
    \centering
    \begin{BVerbatim}
% To see the (page)frames used for hyphenation determination
% \usepackage{showframe}
    \end{BVerbatim}
    \tsCaptionLabelFigure{Preamble -- Package showframe}
\end{figure}

by removing the comment sign (\%). This will display frames in the document
indicating the text areas -- and you can clearly see the overlong lines. The
affected words at the end of the line can then be entered in the
\tsTextItalic{TSHyphenation.tex} file. An entry there follows the following
pattern:

\begin{figure}[H]
    \small
    \centering
    \begin{BVerbatim}
\hyphenation{Thir-ty-Some-thing}
    \end{BVerbatim}
    \tsCaptionLabelFigure{Exemplary content of TSHyphenation.tex}
\end{figure}

An entry corresponds to the word to be separated with a dash after each
syllable. If necessary in the different cases, then the words are separated
by a space. In addition, the following parameter can be changed in the document
\tsTextItalic{TSTemplate.tex}:

\begin{figure}[H]
    \small
    \centering
    \begin{BVerbatim}
% Setting tolerance for word wrap
% See also: https://texfaq.org/FAQ-overfull
\tolerance=3000
    \end{BVerbatim}
    \tsCaptionLabelFigure{Influencing the hyphenation}
\end{figure}

The permissible values are in the range from 0 to 10000.

\section{Commands}
\label{sec:Commands}

To simplify things for newcomers, various commands have been defined. The use
of the commands is optional. However, using them makes it clearer what the
author wants to achieve with the corresponding command. It is recommended to
end commands without parameters with \tsTextMonospace{\{\}}, even if this is
not necessary. However, it leads to the fact that, for example, after such a
command there can also be a space character -- otherwise this is removed by
\LaTeX{} when compiling.

\begin{footnotesize}
    \renewcommand*{\arraystretch}{1.5}
    \begin{longtable}{ | P{0.55\textwidth} | p{0.4\textwidth} | }
        \hline
        \tsTextBold{Command definition}                                                                    &
        \tsTextBold{Meaning}                                                                                 \\
        \hline
        % ----------------------------------------------------------------------
        \tsTextMonospace{\tsBackslash{}tsAppendixSection\{Appendix\}}                                      &
        This command will add an appendix, also a so-called label. Thus the
        appendix can be referenced via
        \tsTextMonospace{\tsBackslash{}ref\{appendix:Appendix\}}
        or via
        \tsTextMonospace{\tsBackslash{}nameref\{appendix:Appendix\}}
        somewhere else.                                                                                      \\
        \hline
        % ----------------------------------------------------------------------
        \tsTextMonospace{\tsBackslash{}tsArrowRight\{\}}                                                   &
        \tsArrowRight{}                                                                                      \\
        \hline
        % ----------------------------------------------------------------------
        \tsTextMonospace{\tsBackslash{}tsArrowRightDouble\{\}}                                             &
        \tsArrowRightDouble{}                                                                                \\
        \hline
        % ----------------------------------------------------------------------
        \tsTextMonospace{\tsBackslash{}tsArrowDown\{\}}                                                    &
        \tsArrowDown{}                                                                                       \\
        \hline
        % ----------------------------------------------------------------------
        \tsTextMonospace{\tsBackslash{}tsArrowDownDouble\{\}}                                              &
        \tsArrowDownDouble{}                                                                                 \\
        \hline
        % ----------------------------------------------------------------------
        \tsTextMonospace{\tsBackslash{}tsArrowLeft\{\}}                                                    &
        \tsArrowLeft{}                                                                                       \\
        \hline
        % ----------------------------------------------------------------------
        \tsTextMonospace{\tsBackslash{}tsArrowLeftDouble\{\}}                                              &
        \tsArrowLeftDouble{}                                                                                 \\
        \hline
        % ----------------------------------------------------------------------
        \tsTextMonospace{\tsBackslash{}tsArrowUp\{\}}                                                      &
        \tsArrowUp{}                                                                                         \\
        \hline
        % ----------------------------------------------------------------------
        \tsTextMonospace{\tsBackslash{}tsArrowUpDouble\{\}}                                                &
        \tsArrowUpDouble{}                                                                                   \\
        \hline
        % ----------------------------------------------------------------------
        \tsTextMonospace{\tsBackslash{}tsBackslash\{\}}                                                    &
        \tsBackslash{}                                                                                       \\
        \hline
        % ----------------------------------------------------------------------
        \tsTextMonospace{\tsBackslash{}tsBullet\{\}}                                                       &
        \tsBullet{}                                                                                          \\
        \hline
        % ----------------------------------------------------------------------
        \tsTextMonospace{\tsBackslash{}tsCaptionLabelFigure\{Figure\}}                                     &
        This command is usually used below a mapping. There is not only a
        description added, but also a so-called label. Thus the figure can be
        referenced via
        \tsTextMonospace{\tsBackslash{}ref\{fig:Figure\}}
        or via
        \tsTextMonospace{\tsBackslash{}nameref\{fig:Figure\}}
        somewhere else.                                                                                      \\
        \hline
        % ----------------------------------------------------------------------
        \tsTextMonospace{\tsBackslash{}tsCaptionLabelFormula\{Formula\}}                                   &
        This command is similar to
        \tsTextMonospace{\tsBackslash{}tsCaptionLabelFigure},
        but is referenced via
        \tsTextMonospace{\tsBackslash{}ref\{for:Formula\}} or via
        \tsTextMonospace{\tsBackslash{}nameref\{for:Formula\}}.                                              \\
        \hline
        % ----------------------------------------------------------------------
        \tsTextMonospace{\tsBackslash{}tsCaptionLabelTable\{Table\}}                                       &
        This command is similar to
        \tsTextMonospace{\tsBackslash{}tsCaptionLabelFigure}, but is
        referenced via
        \tsTextMonospace{\tsBackslash{}ref\{tab:Table\}} or via
        \tsTextMonospace{\tsBackslash{}nameref\{tab:Table\}}.                                                \\
        \hline
        % ----------------------------------------------------------------------
        \tsTextMonospace{\tsBackslash{}tsDegree\{zahl\}}                                                   &
        Without number \tsDegree{} or with number \tsDegree{715}!                                            \\
        \hline
        % ----------------------------------------------------------------------
        \tsTextMonospace{\tsBackslash{}tsFootnoteDef\{text\}\{ref\}}                                       &
        Defines a footnote\tsFootnoteDef{I am a footnote.}{fndef} with
        a label, which can be referenced elsewhere using
        \tsTextMonospace{\tsBackslash{}tsFootnoteRef\{ref\}}.                                                \\
        \hline
        % ----------------------------------------------------------------------
        \tsTextMonospace{\tsBackslash{}tsFootnoteRef\{ref\}}                                               &
        References a footnote\tsFootnoteRef{fndef} which was defined elsewhere
        using \tsTextMonospace{\tsBackslash{}tsFootnoteDef\{text\}\{ref\}}.                                  \\
        \hline
        % ----------------------------------------------------------------------
        \tsTextMonospace{\tsBackslash{}tsImage\{filename\}\{description\}\newline\{options\}}              &
        With this, an image can be included -- it then automatically gets a
        description. It is mandatory to specify three parameter brackets.
        The first parameter is the file name. The second parameter is the
        signature under the image. The third parameter, the options, are
        optional, that is, the brackets can be empty. These are the same
        parameters that can be used with
        \tsTextMonospace{\tsBackslash{}includegraphics},
        e.g. \tsTextMonospace{width=380px}.                                                                  \\
        \hline
        % ----------------------------------------------------------------------
        \tsTextMonospace{\tsBackslash{}tsImageF\{filename\}\{description\}\newline\{options\}}             &
        Like \tsTextMonospace{\tsBackslash{}tsImage}, but with a frame.                                      \\
        \hline
        % ----------------------------------------------------------------------
        \tsTextMonospace{\tsBackslash{}tsImageO\{filename\}\{description\}\newline\{overlay\}\{options\}}  &
        Like \tsTextMonospace{\tsBackslash{}tsImage}, but with an overlaid text.                             \\
        \hline
        % ----------------------------------------------------------------------
        \tsTextMonospace{\tsBackslash{}tsImageOF\{filename\}\{description\}\newline\{overlay\}\{options\}} &
        Like \tsTextMonospace{\tsBackslash{}tsImageF}, but with an overlaid text.                            \\
        \hline
        % ----------------------------------------------------------------------
        \begin{BVerbatim}
\begin{tsLTItemize}
    \item Item I of enumeration list.
    \item Item B of enumeration list.
    \item Item 3 of enumeration list.
\end{tsLTItemize}
        \end{BVerbatim}
                                                                                                           &
        An enumeration list
        \begin{tsLTItemize}
            \item Item I of enumeration list.
            \item Item B of enumeration list.
            \item Item 3 of enumeration list.
        \end{tsLTItemize}
        \\
        \hline
        % ----------------------------------------------------------------------
        \tsTextMonospace{\tsBackslash{}tsTextBold\{text\}}                                                 &
        \tsTextBold{text}                                                                                    \\
        \hline
        % ----------------------------------------------------------------------
        \tsTextMonospace{\tsBackslash{}tsTextItalic\{text\}}                                               &
        \tsTextItalic{text}                                                                                  \\
        \hline
        % ----------------------------------------------------------------------
        \tsTextMonospace{\tsBackslash{}tsTextLower\{text\}}                                                &
        \tsTextLower{text}                                                                                   \\
        \hline
        % ----------------------------------------------------------------------
        \tsTextMonospace{\tsBackslash{}tsTextMonospace\{text\}}                                            &
        \tsTextMonospace{text}                                                                               \\
        \hline
        % ----------------------------------------------------------------------
        \tsTextMonospace{\tsBackslash{}tsTextQuoteF\{text\}}                                               &
        \tsTextQuoteF{text}                                                                                  \\
        \hline
        % ----------------------------------------------------------------------
        \tsTextMonospace{\tsBackslash{}tsTextQuoteG\{text\}}                                               &
        \tsTextQuoteG{text}                                                                                  \\
        \hline
        % ----------------------------------------------------------------------
        \tsTextMonospace{\tsBackslash{}tsTextStrikethrough\{text\}}                                        &
        \tsTextStrikethrough{text}                                                                           \\
        \hline
        % ----------------------------------------------------------------------
        \tsTextMonospace{\tsBackslash{}tsTextUnderline\{text\}}                                            &
        \tsTextUnderline{text}                                                                               \\
        \hline
        % ----------------------------------------------------------------------
        \tsTextMonospace{\tsBackslash{}tsTextUpper\{text\}}                                                &
        \tsTextUpper{text}                                                                                   \\
        \hline
        % ----------------------------------------------------------------------
        \tsCaptionLabelTable{Available commands}
    \end{longtable}
\end{footnotesize}

A look into the source code of this document shows the usage of the commands.

\subsection{Examples to images}

\begin{figure}[H]
    \small
    \centering
    \begin{BVerbatim}
% Insert image TSLogo.png
% File is in subdirectory Images
% Signature is 'Image without parameters'
% Optional parameters are empty
\tsImage{./Images/TSLogo.png}{Image without parameters}{}
    \end{BVerbatim}
    \tsCaptionLabelFigure{Code of image without parameters}
\end{figure}

\tsImage{./Images/TSLogo.png}{Image without parameters}{}

\begin{figure}[H]
    \small
    \centering
    \begin{BVerbatim}
% Insert image TSLogo.png
% File is in subdirectory Images
% Signature is 'Image with parameter'
% Parameter: width=0.5\textwidth => scale the image to half page width
\tsImage{./Images/TSLogo.png}{Image with parameter}{width=0.5\textwidth}
    \end{BVerbatim}
    \tsCaptionLabelFigure{Code of image with parameter}
\end{figure}

\tsImage{./Images/TSLogo.png}{Image with parameter}{width=0.5\textwidth}

\begin{figure}[H]
    \small
    \centering
    \begin{BVerbatim}
% Insert image TSLogo.png with frame
% File is in subdirectory Images
% Signature is 'Image without parameters, with frame'
% Optional parameters are empty
\tsImageF{./Images/TSLogo.png}{Image without parameters, with frame}{}
    \end{BVerbatim}
    \tsCaptionLabelFigure{Code of image without parameters, with frame}
\end{figure}

\tsImageF{./Images/TSLogo.png}{Image without parameters, with frame}{}

\begin{figure}[H]
    \small
    \centering
    \begin{BVerbatim}
% Insert image Garden.jpg with frame
% File is in subdirectory Images
% Signature is 'Image with frame, overlay and parameters'
% Optional parameters are empty
\tsImageO{./Images/Garden.jpg}{Image with frame, overlay and parameters}
    {\tsTextUpper{(C)} 2008, ThirtySomething}{width=0.9\textwidth}
    \end{BVerbatim}
    \tsCaptionLabelFigure{Code image with frame, overlay and parameters}
\end{figure}

\tsImageO{./Images/Garden.jpg}{Image with frame, overlay and parameters}{\tsTextUpper{(C)} 2008, ThirtySomething}{{width=0.9\textwidth}}

\section{Font sizes}
\label{sec:Font sizes}

There may be a need to adjust the font size. For this purpose \LaTeX{} provides
the following options:

\begin{footnotesize}
    \renewcommand*{\arraystretch}{3}
    \begin{longtable}{ | p{0.45\textwidth} | p{0.5\textwidth} | }
        \hline
        \tsTextBold{Command}                             & \tsTextBold{Example}              \\
        \hline
        % ----------------------------------------------------------------------
        \tsBackslash{}Huge\{Huge\}                       & \Huge{Huge}                       \\
        \hline
        % ----------------------------------------------------------------------
        \tsBackslash{}huge\{huge\}                       & \huge{huge}                       \\
        \hline
        % ----------------------------------------------------------------------
        \tsBackslash{}LARGE\{LARGE\}                     & \LARGE{LARGE}                     \\
        \hline
        % ----------------------------------------------------------------------
        \tsBackslash{}Large\{Large\}                     & \Large{Large}                     \\
        \hline
        % ----------------------------------------------------------------------
        \tsBackslash{}normalsize\{normalsize\} (Default) & \normalsize{normalsize (Default)} \\
        \hline
        % ----------------------------------------------------------------------
        \tsBackslash{}small\{small\}                     & \small{small}                     \\
        \hline
        % ----------------------------------------------------------------------
        \tsBackslash{}footnotesize\{footnotesize\}       & \footnotesize{footnotesize}       \\
        \hline
        % ----------------------------------------------------------------------
        \tsBackslash{}scriptsize\{scriptsize\}           & \scriptsize{scriptsize}           \\
        \hline
        % ----------------------------------------------------------------------
        \tsBackslash{}tiny\{tiny\}                       & \tiny{tiny}                       \\
        \hline
        % ----------------------------------------------------------------------
        \tsCaptionLabelTable{Font sizes}
    \end{longtable}
\end{footnotesize}

\section{Tables}

Font sizes play a role in the use of tables when it comes to distributing
the content appropriately among the columns. Here is a sample code how to
make a table.

\begin{figure}[H]
    \small
    \centering
    \begin{BVerbatim}
\begin{small}
    \renewcommand*{\arraystretch}{3}
    \begin{longtable}{ | p{0.45\textwidth} | p{0.5\textwidth} | }
        \hline
        \tsTextBold{Command}       & \tsTextBold{Sample} \\
        \hline
        \tsBackslash{}Huge\{text\} & \Huge{Huge}         \\
        \hline
        \tsCaptionLabelTable{Font sizes}
    \end{longtable}
\end{small}
    \end{BVerbatim}
    \tsCaptionLabelFigure{Tables}
\end{figure}

It is started with a block in one of the desired \nameref{sec:Font sizes}.
The first command within the font environment changes the spacing between the
text lines. If you do not do this, the table content and the table grid will
overlap or the spacing will be very tight. A good value is

\begin{figure}[H]
    \small
    \centering
    \begin{BVerbatim}
\renewcommand*{\arraystretch}{1.5}
    \end{BVerbatim}
    \tsCaptionLabelFigure{Row height in tables}
\end{figure}

The definition of the columns and their widths follows. The total width
available to a text in a line is \tsTextMonospace{\tsBackslash{}textwidth}. Due
to the lines of the table, not quite \tsTextMonospace{1.0 \tsBackslash{}textwidth}
is available. So the sum of all individual column widths should not exceed
\tsTextMonospace{0.9\tsBackslash{}textwidth}.
\bigbreak

To slightly separate column headings from the column contents, the\linebreak
\tsTextMonospace{\tsBackslash{}tsTextBold\{column name\}} command can be used. Each
column value is separated from the next column by a \tsTextMonospace{\&}. The end
of the row is marked with a \tsTextMonospace{\tsBackslash{}\tsBackslash{}}.
\bigbreak

For horizontal lines, a \tsTextMonospace{\tsBackslash{}hline} is inserted.
\bigbreak

The data lines follow the same structure.
\bigbreak

To give the table a referencable name, add a
\tsTextMonospace{\tsBackslash{}tsCaptionLabelTable\{Font sizes\}} at the end.
Check the source of this document in the section \nameref{sec:Font sizes}
to see how it works.
\bigbreak

When using a \tsTextMonospace{verbatim} environment in a table, then the
column type of the table \tsTextBold{must be} of type \tsTextMonospace{P}!
Also important is that the column separator \tsTextMonospace{\&} must be on
the next line after the \tsTextMonospace{\tsBackslash{}end\{tsLTItemize\}}.

\section{The verbatim environment}

With the Verbatim environment text can be displayed in a
\tsTextMonospace{Monospace} font. This is used to display source code.

\begin{figure}[H]
    \centering
    \begin{verbbox}
\begin{figure}[H]
    \small
    \centering
    \begin{BVerbatim}
Sample code of verbatim environment.
    \end{BVerbatim}
    \tsCaptionLabelFigure{Verbatim Environment Demo}
\end{figure}
    \end{verbbox}
    \theverbbox
    \tsCaptionLabelFigure{Verbatim Environment Code}
\end{figure}

In order for the verbatim environment to get a signature, it is embedded in a
figure environment. The \tsTextMonospace{[H]} behind it makes sure that the
environment stays at the desired place in the text and does not end up
somewhere in the document.
\bigbreak

The adjustment of the font size is done as needed. A possible centering  is
achieved here using \tsTextMonospace{\tsBackslash{}centering}.
\bigbreak

\tsTextBold{Note:} If the verbatim environment is to be centered, the text that
is between \tsTextMonospace{\tsBackslash{}begin\{BVerbatim\}} and
\tsTextMonospace{\tsBackslash{}end\{BVerbatim\}} must be leftmost aligned in the
source code of the document. Otherwise the spaces will be part of the
alignment.
\bigbreak

The example from above then produces this result:

\begin{figure}[H]
    \small
    \centering
    \begin{BVerbatim}
Sample code of verbatim environment.
    \end{BVerbatim}
    \tsCaptionLabelFigure{Verbatim Environment Demo}
\end{figure}

\section{Appendices}

In case you want to have appendices, there are two steps. The first one is
required to enable the appendices, the second one is the appendix itself.
For enabling remove the comment sign in the \tsTextItalic{TSTemplate-Meta.tex}
file:

\begin{figure}[H]
    \small
    \centering
    \begin{BVerbatim}
% Switch to enable appendix
\newif\ifUseTsAppendix
\UseTsAppendixtrue
    \end{BVerbatim}
    \tsCaptionLabelFigure{Enabling appendices}
\end{figure}

The second one is the usage of an appendix. You can have a look at the sources
in \tsTextItalic{TSAppendix.tex} or use this example:

\begin{figure}[H]
    \small
    \centering
    \begin{BVerbatim}
\tsAppendixSection{README.md}
    \end{BVerbatim}
    \tsCaptionLabelFigure{Usage of an appendix}
\end{figure}

In this example the appendix \tsTextMonospace{README.md} is added and can be
referenced either by \ref{appendix:README.md} and/or \nameref{appendix:README.md}.
