%% -----------------------------------------------------------------------------
%% MIT License
%%
%% Copyright 2022 ThirtySomething
%%
%% Permission is hereby granted, free of charge, to any person obtaining a
%% copy of this software and associated documentation files (the "Software"),
%% to deal in the Software without restriction, including without limitation
%% the rights to use, copy, modify, merge, publish, distribute, sublicense,
%% and/or sell copies of the Software, and to permit persons to whom the
%% Software is furnished to do so, subject to the following conditions:
%%
%% The above copyright notice and this permission notice shall be included
%% in all copies or substantial portions of the Software.
%%
%% THE SOFTWARE IS PROVIDED "AS IS", WITHOUT WARRANTY OF ANY KIND, EXPRESS
%% OR IMPLIED, INCLUDING BUT NOT LIMITED TO THE WARRANTIES OF MERCHANTABILITY,
%% FITNESS FOR A PARTICULAR PURPOSE AND NONINFRINGEMENT. IN NO EVENT SHALL
%% THE AUTHORS OR COPYRIGHT HOLDERS BE LIABLE FOR ANY CLAIM, DAMAGES OR OTHER
%% LIABILITY, WHETHER IN AN ACTION OF CONTRACT, TORT OR OTHERWISE, ARISING
%% FROM, OUT OF OR IN CONNECTION WITH THE SOFTWARE OR THE USE OR OTHER
%% DEALINGS IN THE SOFTWARE.
%% -----------------------------------------------------------------------------

%% -----------------------------------------------------------------------------
%% Content definition of Latex template
%% -----------------------------------------------------------------------------

\section{License information}

This template is published under the \href{https://opensource.org/licenses/mit-license.php}{MIT license}:

\begin{figure}[H]
    \scriptsize
    \VerbatimInput{LICENSE}
    \tsCaptionLabelFigure{MIT License}
\end{figure}

\section{Introduction}

This is a template for use with \LaTeX{}. An attempt has been made to make
this template as simple as possible for the user.

\section{Template}
\label{sec:Template}

The template is divided into several files, which are listed here below:

\begin{small}
    \renewcommand*{\arraystretch}{1.5}
    \begin{longtable}{ | p{0.45\textwidth} | p{0.5\textwidth} | }
        \hline
        \tsFontBold{File}    & \tsFontBold{Meaning}            \\
        \hline
        % ----------------------------------------------------------------------
        LICENSE              & MIT license file.               \\
        \hline
        % ----------------------------------------------------------------------
        README.md            & Information about the template. \\
        \hline
        % ----------------------------------------------------------------------
        TSAppendix.tex       & The appendices of the document. \\
        \hline
        % ----------------------------------------------------------------------
        TSCommands.tex       & The common predefined commands. \\
        \hline
        % ----------------------------------------------------------------------
        TSContent.tex        & The content of the document.    \\
        \hline
        % ----------------------------------------------------------------------
        TSCustomCommands.tex & The document specific commands. \\
        \hline
        % ----------------------------------------------------------------------
        TSGlossary.tex       & The glossary of the document.   \\
        \hline
        % ----------------------------------------------------------------------
        TSHistory.tex        & Document history.               \\
        \hline
        % ----------------------------------------------------------------------
        TSHyphenation.tex    & Document-specific hyphenation.  \\
        \hline
        % ----------------------------------------------------------------------
        TSTemplate-Meta.tex  & The metadata of the document.   \\
        \hline
        % ----------------------------------------------------------------------
        TSTemplate.pdf       & The result as a PDF file.       \\
        \hline
        % ----------------------------------------------------------------------
        TSTemplate.tex       & The framework document.         \\
        \hline
        % ----------------------------------------------------------------------
        TSTitle.tex          & The title page of the document. \\
        \hline
        % ----------------------------------------------------------------------
        Images/TSLogo.png    & The logo used.                  \\
        \hline
        % ----------------------------------------------------------------------
        \tsCaptionLabelTable{The files of the template}
    \end{longtable}
\end{small}

Custom commands and formatting have been defined for the template. They all
have in common that they start with a \tsFontItalic{\tsBackslash{}ts\ldots{}}.
This serves to distinguish them from the commands otherwise used in \LaTeX{}.
The available commands are explained in the \nameref{sec:Commands} chapter.
\bigbreak

\tsFontBold{NOTE:} The file containing the meta data is depending on the name
of the envelope document \tsFontCode{TSTemplate.tex}. The rule is just simple:
The name will be \tsFontCode{TSTemplate} followed by \tsFontCode{-Meta.tex}.
If you name your document \tsFontCode{MyDocument.tex} then the filename of the
metadata file will be \tsFontCode{MyDocument-Meta.tex}.

\section{Usage}

As already described, the use of the template should be simple. So the user
copies the template, that is, the entire directory with all files and
subdirectories.
\bigbreak

Normally the user adjusts the metadata (file \tsFontItalic{TSTemplate-Meta.tex},
see \nameref{subsec:Metadata}) and the content in the file
\tsFontItalic{TSContent.tex} according to his needs. The rest is controlled
by the template.
\bigbreak

It is recommended to rename the main template file \tsFontItalic{TSTemplate.tex}
into a more more descriptive name.

\tsImageF{./Images/Otherpage.png}{Document page}{height=0.5\textheight}

In principle, nothing needs to be changed in terms of content. Possibly a
list of figures, a list of tables or a glossary is not desired. These are
enabled in the framework document \tsFontItalic{TSTemplate.tex}. To disable
them, place the comment character (\%) in the corresponding lines. The places
are easy to find. Firstly, they are located quite at the end of the template
and secondly, they are marked with appropriate comments:

\begin{figure}[H]
    \scriptsize
    \centering
    \begin{BVerbatim}
% Print list of figures (lof)
\newpage
\listoffigures

% Print list of table (lot)
\newpage
\listoftables

% Print glossary
\newpage
\printglossaries
    \end{BVerbatim}
    \tsCaptionLabelFigure{Indexes}
\end{figure}

If one (or more) of the options is disabled, it will be automatically removed
from the table of contents.

\subsection{Metadata I}
\label{subsec:Metadata-I}

There is metadata for each document. This means the following definitions at
the moment:

\begin{footnotesize}
    \renewcommand*{\arraystretch}{1.5}
    \begin{longtable}{ | p{0.45\textwidth} | p{0.5\textwidth} | }
        \hline
        \tsFontBold{Metadefinition}       & \tsFontBold{Meaning}                                     \\
        \hline
        % ----------------------------------------------------------------------
        \tsBackslash{}tsAuthor\{\}        & The author of the document.                              \\
        \hline
        % ----------------------------------------------------------------------
        \tsBackslash{}tsSubject\{\}       & The subject of the document.                             \\
        \hline
        % ----------------------------------------------------------------------
        \tsBackslash{}tsTitle\{\}         & The title of the document.                               \\
        \hline
        % ----------------------------------------------------------------------
        \tsBackslash{}tsURL\{\}           & The url of the author.                                   \\
        \hline
        % ----------------------------------------------------------------------
        \tsBackslash{}tsVersionMajor\{\}  & The major version of the document.\tsFootnoteDef{See
        \href{https://semver.org}{Semantic Versioning} for more details.}{semver}                    \\
        \hline
        % ----------------------------------------------------------------------
        \tsBackslash{}tsVersionMinor\{\}  & The minor version of the document.\tsFootnoteRef{semver} \\
        \hline
        % ----------------------------------------------------------------------
        \tsBackslash{}tsVersionPatch\{\}  & The patch number of the document.\tsFootnoteRef{semver}  \\
        \hline
        % ----------------------------------------------------------------------
        \tsBackslash{}tsVersionString\{\} & Complete version string.\tsFootnoteRef{semver}           \\
        \hline
        % ----------------------------------------------------------------------
        \tsCaptionLabelTable{Metadata I}
    \end{longtable}
\end{footnotesize}

These metadata are referenced in various places in the template. For example,
they are used in the title page of the document. The version number is based
on semantic versioning\tsFootnoteRef{semver}.

\tsImageF{./Images/Titlepage.png}{Title page}{height=0.5\textheight}

They are also used to set metadata in the generated PDF:

\tsImage{./Images/PDF-Properties.png}{PDF properties}{width=0.6\textwidth}

\subsection{Metadata II}
\label{subsec:Metadata-II}

Additional to the metadata of section \ref{subsec:Metadata-I} there exists
metadata to influence the filenames. Except for the \tsFontItalic{TSTemplate-Meta.tex}
file all other files are used indirect by commands. So there are these
commands available:

\begin{footnotesize}
    \renewcommand*{\arraystretch}{1.5}
    \begin{longtable}{ | p{0.45\textwidth} | p{0.5\textwidth} | }
        \hline
        \tsFontBold{Metadefinition}            & \tsFontBold{Meaning}                  \\
        \hline
        % ----------------------------------------------------------------------
        \tsBackslash{}tsAppendixFile\{\}       & Filename for appendices, \newline
        default \tsFontItalic{TSAppendix.tex}                                          \\
        \hline
        % ----------------------------------------------------------------------
        \tsBackslash{}tsCommandsFile\{\}       & Filename for commands, \newline
        default \tsFontItalic{TSCommands.tex}                                          \\
        \hline
        % ----------------------------------------------------------------------
        \tsBackslash{}tsContentFile\{\}        & Filename for content,\newline
        default \tsFontItalic{TSContent.tex}                                           \\
        \hline
        % ----------------------------------------------------------------------
        \tsBackslash{}tsCustomCommandsFile\{\} & Filename for custom commands,\newline
        default \tsFontItalic{TSCustomCommands.tex}                                    \\
        \hline
        % ----------------------------------------------------------------------
        \tsBackslash{}tsGlossaryFile\{\}       & Filename for glossary,\newline
        default \tsFontItalic{TSGlossary.tex}                                          \\
        \hline
        % ----------------------------------------------------------------------
        \tsBackslash{}tsHistoryFile\{\}        & Filename for history,\newline
        default \tsFontItalic{TSHistory.tex}                                           \\
        \hline
        % ----------------------------------------------------------------------
        \tsBackslash{}tsHyphenationFile\{\}    & Filename for hyphenation,\newline
        default \tsFontItalic{TSHyphenation.tex}                                       \\
        \hline
        % ----------------------------------------------------------------------
        \tsBackslash{}tsLogoFile\{\}           & Filename for logo,\newline
        default \tsFontItalic{./Images/TSLogo.png}                                     \\
        \hline
        % ----------------------------------------------------------------------
        \tsBackslash{}tsTitleFile\{\}          & Filename for title,\newline
        default \tsFontItalic{TSTitle.tex}                                             \\
        \hline
        % ----------------------------------------------------------------------
        \tsCaptionLabelTable{Metadata II}
    \end{longtable}
\end{footnotesize}

\subsection{Document content}

The document content is entered like a normal \LaTeX{} document. For this the
file \tsFontItalic{TSContent.tex} is used, in which the user enters the text.
The basic conditions such as the inclusion of packages are regulated in the file
\tsFontItalic{TSTemplate.tex}.

\subsection{Glossary}

The user has the option of entering his own glossary. The entries for the
glossary are made in the file \tsFontItalic{TSGlossary.tex}. \gls{TS}
is entered here as an example. This looks like this:

\begin{figure}[H]
    \scriptsize
    \centering
    \begin{BVerbatim}
\newglossaryentry{TS}{name={ThirtySomething},description={A glossary entry}}
    \end{BVerbatim}
    \tsCaptionLabelFigure{Definition of a glossary entry}
\end{figure}

The values have the following correspondences:

\begin{itemize}
    \item \tsFontCode{TS} -- This is the abbreviation used to reference the
          keyword in the document.
    \item \tsFontCode{ThirtySomething} -- This is the keyword listed in the
          glossary.
    \item \tsFontCode{Ein Glossareintrag} -- This is the description for the
          keyword listed in the glossary.
\end{itemize}

The glossary then lists the keyword, the description and the page number(s)
where the keyword was used everywhere. A prerequisite for this is, of course,
the use of the command \tsFontCode{\tsBackslash{}gls\{abbreviation\}} with the
existing abbreviation. In the above example, the usage looks like this:

\begin{figure}[H]
    \small
    \centering
    \begin{BVerbatim}
\gls{TS}
    \end{BVerbatim}
    \tsCaptionLabelFigure{Use of a glossary entry}
\end{figure}

In case a glossary is to be used, the document must be translated with
\tsFontCode{makeglossaries}.

\begin{figure}[H]
    \small
    \centering
    \begin{BVerbatim}
pdflatex
makeglossaries
pdflatex
pdflatex
    \end{BVerbatim}
    \tsCaptionLabelFigure{Translate with glossary}
\end{figure}

\subsection{Hyphenation}

In some circumstances, hyphenation is not performed for some words. This may
be because these words are very specific and \LaTeX{} does not know them and
therefore cannot hyphenate them. The result may be overlong lines, which are
clearly visible in most cases. In some cases this overlength is not
immediately obvious. For this you can temporarily add the line

\begin{figure}[H]
    \small
    \centering
    \begin{BVerbatim}
% To see the (page)frames used for hyphenation determination
% \usepackage{showframe}
    \end{BVerbatim}
    \tsCaptionLabelFigure{Preamble -- Package showframe}
\end{figure}

by removing the comment sign (\%). This will display frames in the document
indicating the text areas -- and you can clearly see the overlong lines. The
affected words at the end of the line can then be entered in the
\tsFontItalic{TSHyphenation.tex} file. An entry there follows the following
pattern:

\begin{figure}[H]
    \small
    \centering
    \begin{BVerbatim}
\hyphenation{Thir-ty-Some-thing}
    \end{BVerbatim}
    \tsCaptionLabelFigure{Exemplary content of TSHyphenation.tex}
\end{figure}

An entry corresponds to the word to be separated with a separator after each
syllable. If necessary in the different cases, then the words are separated
by a space. In addition, the following parameter can be changed in the document
\tsFontItalic{TSTemplate.tex}:

\begin{figure}[H]
    \small
    \centering
    \begin{BVerbatim}
% Setting tolerance for word wrap
% See also: https://texfaq.org/FAQ-overfull
\tolerance=3000
    \end{BVerbatim}
    \tsCaptionLabelFigure{Influencing the hyphenation}
\end{figure}

The permissible values are in the range from 0 to 10000. setting, hyphenation
is performed earlier or later.

\section{Commands}
\label{sec:Commands}

To simplify things for newcomers, various commands have been defined. The use
of the commands is optional. However, using them makes it clearer what the
author wants to achieve with the corresponding command. It is recommended to
end commands without parameters with \{\}, even if this is not necessary.
However, it leads to the fact that, for example, after such a command there
can also be a space character -- otherwise this is removed by \LaTeX{} when
compiling.

\begin{footnotesize}
    \renewcommand*{\arraystretch}{1.5}
    \begin{longtable}{ | p{0.5\textwidth} | p{0.45\textwidth} | }
        \hline
        \tsFontBold{Command definition}                                                             & \tsFontBold{Meaning}                                      \\
        \hline
        % ----------------------------------------------------------------------
        \tsBackslash{}tsAppendixSection\{Appendix\}                                                 & This command will add an appendix,
        also a so-called label. Thus the figure can be referenced via
        \tsBackslash{}ref\{appendix:Appendix\} or via \tsBackslash{}nameref\{appendix:Appendix\} in the text.                                                   \\
        \hline
        % ----------------------------------------------------------------------
        \tsBackslash{}tsArrowRight\{\}                                                              & \tsBackslash{}tsArrowRight\{\} draws an arrow
        right (\tsArrowRight{}).                                                                                                                                \\
        \hline
        % ----------------------------------------------------------------------
        \tsBackslash{}tsArrowRightDouble\{\}                                                        & \tsBackslash{}tsArrowRightDouble\{\} draws a double arrow
        right (\tsArrowRightDouble{}).                                                                                                                          \\
        \hline
        % ----------------------------------------------------------------------
        \tsBackslash{}tsArrowDown\{\}                                                               & \tsBackslash{}tsArrowDown\{\} draws an arrow
        down (\tsArrowDown{}).                                                                                                                                  \\
        \hline
        % ----------------------------------------------------------------------
        \tsBackslash{}tsArrowDownDouble\{\}                                                         & \tsBackslash{}tsArrowDownDouble\{\} draws a double arrow
        down (\tsArrowDownDouble{}).                                                                                                                            \\
        \hline
        % ----------------------------------------------------------------------
        \tsBackslash{}tsArrowLeft\{\}                                                               & \tsBackslash{}tsArrowLeft\{\} draws an arrow
        left (\tsArrowLeft{}).                                                                                                                                  \\
        \hline
        % ----------------------------------------------------------------------
        \tsBackslash{}tsArrowLeftDouble\{\}                                                         & \tsBackslash{}tsArrowLeftDouble\{\} draws a double arrow
        left (\tsArrowLeftDouble{}).                                                                                                                            \\
        \hline
        % ----------------------------------------------------------------------
        \tsBackslash{}tsArrowUp\{\}                                                                 & \tsBackslash{}tsArrowUp\{\} draws an arrow
        up (\tsArrowUp{}).                                                                                                                                      \\
        \hline
        % ----------------------------------------------------------------------
        \tsBackslash{}tsArrowUpDouble\{\}                                                           & \tsBackslash{}tsArrowUpDouble\{\} draws a double arrow
        up (\tsArrowUpDouble{}).                                                                                                                                \\
        \hline
        % ----------------------------------------------------------------------
        \tsBackslash{}tsBackslash\{\}                                                               & \tsBackslash{}tsBackslash\{\} prints a backslash
        (\tsBackslash{}).                                                                                                                                       \\
        \hline
        % ----------------------------------------------------------------------
        \tsBackslash{}tsBullet\{\}                                                                  & With this command you can to insert a bullet
        (\tsBullet{}).                                                                                                                                          \\
        \hline
        % ----------------------------------------------------------------------
        \tsBackslash{}tsCaptionLabelFigure\{Figure\}                                                & This command is usually used below a mapping.
        It not only adds a description, but also a so-called label. Thus the figure can be referenced via
        \tsBackslash{}ref\{fig:Figure\} or via \tsBackslash{}nameref\{fig:Figure\} in the text.                                                                 \\
        \hline
        % ----------------------------------------------------------------------
        \tsBackslash{}tsCaptionLabelFormula\{Formula\}                                              & This command is similar to
        \tsBackslash{}tsCaptionLabelFigure, but is referenced via \tsBackslash{}ref\{for:Formula\} or via
        \tsBackslash{}nameref\{for:Formula\}.                                                                                                                   \\
        \hline
        % ----------------------------------------------------------------------
        \tsBackslash{}tsCaptionLabelTable\{Table\}                                                  & This command is similar to
        \tsBackslash{}tsCaptionLabelFigure, but is referenced via \tsBackslash{}ref\{tab:Table\} or via
        \tsBackslash{}nameref\{tab:Table\}.                                                                                                                     \\
        \hline
        % ----------------------------------------------------------------------
        \tsBackslash{}tsDegree\{zahl\}                                                              & Output of the degree sign with an optional
        number. Without number \tsDegree{} or with number \tsDegree{715}!                                                                                       \\
        \hline
        % ----------------------------------------------------------------------
        \tsBackslash{}tsFontBold\{text\}                                                            & A text is output with bold font.
        \tsFontBold{Bold!}                                                                                                                                      \\
        \hline
        % ----------------------------------------------------------------------
        \tsBackslash{}tsFontCode\{text\}                                                            & A text is output with a monospaced font.
        \tsFontCode{Code!}                                                                                                                                      \\
        \hline
        % ----------------------------------------------------------------------
        \tsBackslash{}tsFontItalic\{text\}                                                          & A text is output with italic font.
        \tsFontItalic{Italic!}                                                                                                                                  \\
        \hline
        % ----------------------------------------------------------------------
        \tsBackslash{}tsFontStrikethrough\{text\}                                                   & A text is striked through.
        \tsFontStrikethrough{Strikethrough!}                                                                                                                    \\
        \hline
        % ----------------------------------------------------------------------
        \tsBackslash{}tsFontUnderline\{text\}                                                       & A text is underlined.
        \tsFontUnderline{Underlined!}                                                                                                                           \\
        \hline
        % ----------------------------------------------------------------------
        \tsBackslash{}tsFootnoteDef\{text\}\{ref\}                                                  & Defines a footnote\tsFootnoteDef{I am a
        footnote.}{fndef} with label, which can be referenced elsewhere with \tsBackslash{}tsFootnoteRef\{ref\}.                                                \\
        \hline
        % ----------------------------------------------------------------------
        \tsBackslash{}tsFootnoteRef\{ref\}                                                          & References a footnote\tsFootnoteRef{fndef}
        which was defined elsewhere with \tsBackslash{}tsFootnoteDef\{text\}\{ref\}.                                                                            \\
        \hline
        % ----------------------------------------------------------------------
        \tsBackslash{}tsImage\{filename\}\{description\}\{options\}                                 & With this, an image can be included - it then
        automatically gets a description. It is mandatory to specify three parameter brackets. The first parameter is
        the file name. The second parameter is the signature under the image. The third parameter, the options, are
        optional, that is, the brackets can be empty. These are the parameters that can be used with
        \tsBackslash{}includegraphics, e.g. \tsFontCode{width=380px}.                                                                                           \\
        \hline
        % ----------------------------------------------------------------------
        \tsBackslash{}tsImageF\{filename\}\{description\}\{options\}                                & Like \tsBackslash{}tsImage, but with a frame.             \\
        \hline
        % ----------------------------------------------------------------------
        \tsBackslash{}begin\{tsLTItemize\} \tsBackslash{}item Enum \tsBackslash{}end\{tsLTItemize\} & An enumeration environment for tables.
        \begin{tsLTItemize}
            \item An enumeration list.
        \end{tsLTItemize}                                                                                                                               \\
        \hline
        % ----------------------------------------------------------------------
        \tsBackslash{}tsTextQuoteF\{\}                                                              & Sets a text in French quotation marks.
        \tsTextQuoteF{Quoted!}                                                                                                                                  \\
        \hline
        % ----------------------------------------------------------------------
        \tsBackslash{}tsTextQuoteG\{\}                                                              & Sets a text in German quotation marks.
        \tsTextQuoteG{Quoted!}                                                                                                                                  \\
        \hline
        % ----------------------------------------------------------------------
        \tsBackslash{}tsTextLower\{text\}                                                           & Sets a text lower. \tsTextLower{Lower!}                   \\
        \hline
        % ----------------------------------------------------------------------
        \tsBackslash{}tsTextUpper\{text\}                                                           & Sets a text raised. \tsTextUpper{Upper!}                  \\
        \hline
        % ----------------------------------------------------------------------
        \tsCaptionLabelTable{Available commands}
    \end{longtable}
\end{footnotesize}

A look into the source code of this document shows the usage of the commands.

\subsection{Examples to images}

\begin{figure}[H]
    \small
    \centering
    \begin{BVerbatim}
% Insert image TSLogo.png
% File is in subdirectory Images
% Signature is 'Image without parameters'
% Optional parameters are empty
\tsImage{./Images/TSLogo.png}{Image without parameters}{}
    \end{BVerbatim}
    \tsCaptionLabelFigure{Code of image without parameters}
\end{figure}

\tsImage{./Images/TSLogo.png}{Image without parameters}{}

\begin{figure}[H]
    \small
    \centering
    \begin{BVerbatim}
% Insert image TSLogo.png
% File is in subdirectory Images
% Signature is 'Image with parameter'
% Parameter: width=0.5\textwidth => scale the image to half page width
\tsImage{./Images/TSLogo.png}{Image with parameter}{width=0.5\textwidth}
    \end{BVerbatim}
    \tsCaptionLabelFigure{Code of image with parameter}
\end{figure}

\tsImage{./Images/TSLogo.png}{Image with parameter}{width=0.5\textwidth}

\begin{figure}[H]
    \small
    \centering
    \begin{BVerbatim}
% Insert image TSLogo.png with frame
% File is in subdirectory Images
% Signature is 'Image without parameters, with frame'
% Optional parameters are empty
\tsImageF{./Images/TSLogo.png}{image without parameters, with frame}{}
    \end{BVerbatim}
    \tsCaptionLabelFigure{Code of image without parameters, with frame}
\end{figure}

\tsImageF{./Images/TSLogo.png}{Image without parameters, with frame}{}

\section{Font sizes}
\label{sec:Font sizes}

There may be a need to adjust the font size. For this purpose \LaTeX{} provides
the following options:

\begin{footnotesize}
    \renewcommand*{\arraystretch}{3}
    \begin{longtable}{ | p{0.45\textwidth} | p{0.5\textwidth} | }
        \hline
        \tsFontBold{Command}                       & \tsFontBold{Example}              \\
        \hline
        % ----------------------------------------------------------------------
        \tsBackslash{}Huge\{text\}                 & \Huge{Huge}                       \\
        \hline
        % ----------------------------------------------------------------------
        \tsBackslash{}huge\{text\}                 & \huge{huge}                       \\
        \hline
        % ----------------------------------------------------------------------
        \tsBackslash{}LARGE\{text\}                & \LARGE{LARGE}                     \\
        \hline
        % ----------------------------------------------------------------------
        \tsBackslash{}Large\{text\}                & \Large{Large}                     \\
        \hline
        % ----------------------------------------------------------------------
        \tsBackslash{}normalsize\{text\} (Default) & \normalsize{normalsize (Default)} \\
        \hline
        % ----------------------------------------------------------------------
        \tsBackslash{}small\{text\}                & \small{small}                     \\
        \hline
        % ----------------------------------------------------------------------
        \tsBackslash{}footnotesize\{text\}         & \footnotesize{footnotesize}       \\
        \hline
        % ----------------------------------------------------------------------
        \tsBackslash{}scriptsize\{text\}           & \scriptsize{scriptsize}           \\
        \hline
        % ----------------------------------------------------------------------
        \tsBackslash{}tiny\{text\}                 & \tiny{tiny}                       \\
        \hline
        % ----------------------------------------------------------------------
        \tsCaptionLabelTable{Font sizes}
    \end{longtable}
\end{footnotesize}

\section{Tables}

Font sizes play a role in the use of tables when it comes to distributing
the content appropriately among the columns. Here is a sample code how to
make a table.

\begin{figure}[H]
    \small
    \centering
    \begin{BVerbatim}
\begin{small}
    \renewcommand*{\arraystretch}{3}
    \begin{longtable}{ | p{0.45\textwidth} | p{0.5\textwidth} | }
        \hline
        \tsFontBold{Command}       & \tsFontBold{Sample} \\
        \hline
        \tsBackslash{}Huge\{text\} & \Huge{Huge}         \\
        \hline
        \tsCaptionLabelTable{Font sizes}
    \end{longtable}
\end{small}
    \end{BVerbatim}
    \tsCaptionLabelFigure{Tables}
\end{figure}

It is started with a block in one of the desired \nameref{sec:Font sizes}.
The first command within the font environment changes the spacing between the
text lines. If you do not do this, the table content and the table grid will
overlap or the spacing will be very tight. A good value is

\begin{figure}[H]
    \small
    \centering
    \begin{BVerbatim}
\renewcommand*{\arraystretch}{1.5}
    \end{BVerbatim}
    \tsCaptionLabelFigure{Row height in tables}
\end{figure}

The definition of the columns and their widths follows. The total width
available to a text in a line is \tsFontCode{\tsBackslash{}textwidth}. Due
to the lines of the table, not quite \tsFontCode{1.0 \tsBackslash{}textwidth}
is available. So the sum of all individual column widths should not exceed
\tsFontCode{0.9\tsBackslash{}textwidth}.
\bigbreak

To slightly separate column headings from the column contents, the\linebreak
\tsFontCode{\tsBackslash{}tsFontBold\{column name\}} command can be used. Each
column value is separated from the next column by a \tsFontCode{\&}. The end
of the row is marked with a \tsFontCode{\tsBackslash{}\tsBackslash{}}.
\bigbreak

For horizontal lines, a \tsFontCode{\tsBackslash{}hline} is inserted.
\bigbreak

The data lines follow the same structure.
\bigbreak

To give the table a referencable name, add a
\tsFontCode{\tsBackslash{}tsCaptionLabelTable\{Font sizes\}} at the end. How
the whole thing looks then, you can look up in the chapter
\nameref{sec:Font sizes} in the source code or here in the document.

\section{The verbatim environment}

With the Verbatim environment text can be displayed in a kind of
\tsFontCode{Monospace} font. This is used to display source code.

\begin{figure}[H]
    \centering
    \begin{verbbox}
        \begin{figure}[H]
            \small
            \centering
            \begin{BVerbatim}
Sample code of verbatim environment.
            \end{BVerbatim}
            \tsCaptionLabelFigure{Verbatim Environment Demo}
        \end{figure}
    \end{verbbox}
    \theverbbox
    \tsCaptionLabelFigure{Verbatim Environment Code}
\end{figure}

In order for the verbatim environment to get a signature, it is embedded in a
figure environment. The \tsFontCode{[H]} behind it makes sure that the
environment stays in the desired place in the text and does not end up
somewhere in the document.
\bigbreak

The adjustment of the font size is done as needed. So is possible centering,
which is achieved here using \tsFontCode{\tsBackslash{}centering}.
\tsFontBold{Note:} If the verbatim environment is to be centered, the text that
is between \tsFontCode{\tsBackslash{}begin\{BVerbatim\}} and
\tsFontCode{\tsBackslash{}end\{BVerbatim\}} must be leftmost aligned in the
source code of the document. Otherwise, the spaces will be included in the
centering.
\bigbreak

The example from above then produces this result:

\begin{figure}[H]
    \small
    \centering
    \begin{BVerbatim}
Sample code of verbatim environment.
    \end{BVerbatim}
    \tsCaptionLabelFigure{Verbatim Environment Demo}
\end{figure}

\section{Appendices}

In case you want to have appendices, there are two steps. The first one is
required to enable the appendices, the second one is the appendix itself.
For enabling remove the comment sign.

\begin{figure}[H]
    \small
    \centering
    \begin{BVerbatim}
% Include appendices
\input{\tsAppendixFile{}}
    \end{BVerbatim}
    \tsCaptionLabelFigure{Enabling appendices}
\end{figure}

The second one is the usage of an appendix. You can have a look at the sources
in \tsFontCode{TSAppendix.tex} or use this example:

\begin{figure}[H]
    \small
    \centering
    \begin{BVerbatim}
\tsAppendixSection{README.md}
    \end{BVerbatim}
    \tsCaptionLabelFigure{Usage of an appendix}
\end{figure}

In this example the appendix \tsFontCode{README.md} is added and can be
referenced either by \ref{appendix:README.md} and/or \nameref{appendix:README.md}.
